% Merci a Yannick Chevallier pour la version mathjax-enabled
\documentclass{article}
\usepackage[utf8]{inputenc}
\usepackage[T1]{fontenc}
\usepackage{amsmath,amsfonts,amssymb}
\usepackage[francais]{babel}
\usepackage{times}
\usepackage{xspace}
\usepackage{makeidx}
\usepackage{ifpdf}
\ifpdf
 \usepackage[colorlinks,pdftex]{hyperref}
\else
 \usepackage[ps2pdf,breaklinks=true,colorlinks=true,linkcolor=red,citecolor=green]{hyperref}
\fi
\newcommand{\bs}{\symbol{92}}
\newtheorem{thm}{Th\'eor\`eme}
\newtheorem{prop}[thm]{Proposition}
\newtheorem{defn}[thm]{D\'efinition}
\makeindex

% \renewcommand{\cuttingunit}{subsection} (add HEVEA after % for hacha)
\input{giacfr.tex}
\ifhevea
\(
\newcommand{\R}{{\mathbb{R}}}
\newcommand{\C}{{\mathbb{C}}}
\newcommand{\Z}{{\mathbb{Z}}}
\newcommand{\N}{{\mathbb{N}}}
\newcommand{\Q}{{\mathbb{Q}}}
\newcommand{\tr}{\mbox{tr\,}}
\)
\else
\newcommand{\R}{{\mathbb{R}}}
\newcommand{\C}{{\mathbb{C}}}
\newcommand{\Z}{{\mathbb{Z}}}
\newcommand{\N}{{\mathbb{N}}}
\newcommand{\Q}{{\mathbb{Q}}}
\newcommand{\tr}{\mbox{tr\,}}
\fi
%HEVEA\htmlfoot{Retour \`a la page principale de \ahref{http://www-fourier.univ-grenoble-alpes.fr/\home{parisse}/giac_fr.html}{Giac/Xcas}.}
\giacmathjax
%HEVEA\renewcommand{\footertext}{}

\begin{document}
\title{$\chi$CAS: Xcas sur calculatrices Casio Graph 90, 35+eii, Math+}
\author{Bernard.Parisse@univ-grenoble-alpes.fr}
\date{2019, 2022, 2025}

\begin{giacjshere}

\maketitle

%% {\bf Demandez à Casio de conserver les addins sur les Graph Math+}:
%% \ahref{mailto:education-france@casio.fr?subject=KhiCAS sur Graph Math&body=Bonjour\%0d\%0aEst\%2dil possible de rendre les Graph Math\%2b compatibles avec KhiCAS\%3f\%0d\%0aMerci d'avance\%21\%0d\%0a}{envoyez un email maintenant} 
%% en personnalisant le contenu du message si possible.\\ \\
%% Casio lance sur le marché à la rentrée 2024 une calculatrice
%% qui remplace la Graph 90. {\bf Mais la Graph Math+ ne permet actuellement pas
%% d'exécuter des addins, ce qui la rend incompatible avec KhiCAS}.
%% Casio a annoncé une version 2 du logiciel pour 2025, il n'y a aucun
%% obstacle technique à réactiver les addins, c'est vraiment une
%% question ``politique''. Mes demandes à Casio de réactiver les addins
%% n'ont pas abouti. Si vous voulez défendre l'accès de tous au
%% calcul formel sur calculatrice, 
%% et pas seulement aux élèves qui peuvent se payer
%% une calculatrice haut de gamme deux fois plus chère, faites-le savoir
%% à Casio {\bf et déconseillez l'achat de la Graph Math+ pour le moment.}
%% En attendant, vous pouvez acheter
%% ou/et conseiller la Graph 90 ou la Graph 35eii chez Casio, ou
%% passer chez Numworks si la facilité d'utilisation est un critère primordial, 
%% tous ces modèles permettent d'utiliser KhiCAS. 
%% C'est à mon avis la meilleure stratégie pour faire bouger les positions
%% chez Casio, qui va forcément regarder les ventes de Math+ à cette 
%% rentrée 2024. 


\tableofcontents

%\printindex

\begin{abstract}
Ce document explique comment prendre en main et utiliser efficacement
sur calculatrices Casio Graph 90+e, 35+eii ou Math+
le syst\`eme de calcul formel et géométrie analytique
$\chi$CAS, une version adapt\'ee
du \ahref{https://www-fourier.univ-grenoble-alpes.fr/\home{parisse}/giac_fr.html}{logiciel Xcas} pour ces calculatrices.

Ce document est interactif, vous pouvez modifier les commande
et voir le r\'esultat de l'ex\'ecution des commandes
propos\'ees en exemple en cliquant sur le bouton \framebox{ok} (ou en
validant avec la touche Entr\'ee).
\end{abstract}

\section{Introduction et installation} \label{sec:install}
$\chi$CAS est une version adapt\'ee du
\ahref{https://www-fourier.univ-grenoble-alpes.fr/\home{parisse}/giac_fr.html}{logiciel Xcas} 
pour calculatrices, ici de la marque Casio. 
KhiCAS est utilisable en mode examen sur les Graph 90 et 35eii.
%% Si vous pensez comme moi qu'il s'agit d'une aberration, faites-le
%% savoir \`a Casio France, plus il y aura de profs qui le demandent,
%% plus il y aura de chances que Casio mette en place un m\'ecanisme
%% pour rendre $\chi$CAS compatible avec le mode examen.

%% Sur les Casio Graph 90 couleur, il existe deux versions, une version
%% light en un seul fichier, et une version plus compl\`ete en 2
%% fichiers. Cette version contient plus de commandes (par exemple des commandes de
%% g\'eom\'etrie), un moteur d'affichage 3d, un tableur formel 
%% et un interpr\'eteur MicroPython un peu plus
%% r\'ecent que celui de Casio, ainsi que des modules plus complets et
%% nombreux que ceux du constructeur. 
%% Ces commandes suppl\'ementaires sont d\'ecrites dans la section \ref{sec:2}.

{\bf Sur les Casio Graph Math+, le clavier a été notablement modifié}, il n'y
a plus de touches F1 à F6 et les 6 menus en bas de l'écran
sont remplacés dans l'OS par 2 onglets de navigation. 
Notez que l'interface de $\chi$CAS ne suit pas cette nouvelle logique,
elle conserve la logique des Casio 90/35. Les touches en correspondance 
des légendes de menu permettent de les activer. La touche correspondant
à F4 étant une touche de déplacement sur la Math+, elle n'a plus de 
légende en menu. 
L'action correspondante à F4 sur la 90, qui est l'accès au menu 
de toutes les commandes, s'effectue avec la touche Tools sur la Math+.
Il n'y a plus de touche MENU sur la Math+, c'est la touche Home
qui joue ce rôle, à la position de la touche F2 sur la 90.
Ainsi pour quitter $\chi$CAS sur la Math+, on tape shift OFF ou on 
tape sur Home et on valide l'entrée de menu \verb|quit| qui apparait.
Il peut être nécessaire de lancer une application Casio avant de relancer
un addin, sinon on risque un crash, sauf si KhiCAS est le seul addin présent.


\subsection{Compatibilité mode examen Graph 35eii et 90e.}
{\bf KhiCAS est conforme à la réglementation du bac en France,
\ahref{https://www.education.gouv.fr/bo/15/Hebdo42/MENS1523092C.htm}{circulaire 2015-178 du 1er octobre 2015}} et
\ahref{http://www-fourier.ujf-grenoble.fr/\home{parisse}/irem/Calculatrices_examens_specifications_techniques.pdf}{cahier des charges},
au même titre que les calculatrices CAS des constructeurs 
(HP Prime, TI Nspire CX/CX2, Casio Classpad), il est donc utilisable
aux épreuves de terminale de maths du bac. 
Mais attention, certains autres
concours ou examens interdisent l'utilisation
de calculatrices formelles. Il est de la responsabilit\'e de l'utilisateur
de v\'erifier que les calculatrices formelles sont autoris\'ees
avant d'utiliser $\chi$CAS dans un autre examen ou concours. Les auteurs
ne sauraient \^etre tenus pour responsables en cas d'utilisation
non autoris\'ee.

Pour utiliser KhiCAS en mode examen, il est nécessaire
d'exécuter (une fois pour toutes) un programme de mise en compatibilité,
utilisable sur les OS 3.80.1 (Graph 90, fonctionne aussi sur OS 3.60) 
ou 3.70 (Graph 35eii, fonctionne aussi sur 3.40). 
Ce programme sécurisé a été écrit par Sébastien Michelland de Planète Casio 
et est disponible depuis le 23 mai 2025. Il n'est pas encore disponible
pour les Graph Math+.

Vérifiez votre version d'OS depuis MENU, Système, F5. 

{\bf Si vous n'avez pas la bonne version},
téléchargez l'installeur de la version compatible de l'OS
\begin{itemize}
\item 
OS 3.80.1 (Graph 90), installeur pour
\ahref{https://tiplanet.org/forum/archives_voir.php?id=4173827}{Windows},
\ahref{https://tiplanet.org/forum/archives_voir.php?id=3338352}{Mac}
ou \ahref{https://tiplanet.org/forum/archives_list.php?order=hit&cat=OS+cprizm&multi_chaine_search=3.60}{OS 3.60, Graph 90}.
%\ahref{https://tiplanet.org/forum/archives_voir.php?id=4173827}{Windows},
%\ahref{https://tiplanet.org/forum/archives_voir.php?id=3338352}{Mac}.
\item
\ahref{https://tiplanet.org/forum/archives_voir.php?id=3471077}{OS 3.70 (Graph 35eii) Windows} ou
\ahref{https://tiplanet.org/forum/archives_voir.php?id=2648557}{OS 3.40 (Graph 35eii) Windows}
\end{itemize}
Lancez le programme de mise à jour sur votre ordinateur, sur la calculatrice,
toujours depuis Système, il faut taper F6 puis F5 (OS Update).

Si le programme d'installation refuse de se lancer parce que
vous avez une version trop récente de l'OS, vérifiez qu'il s'agit bien
de l'installeur de l'OS 3.80.1 pour la Graph 90 ou 3.70 sur la 35eii.
Si le problème d'installation persiste, rendez-vous section
\ref{sec:oserror}.

{\bf Lorsque votre calculatrice a la bonne version de l'OS}, 
récupérez le logiciel de mise en
compatibilité de KhiCAS en mode examen pour Windows~:
\begin{itemize}
\item \ahref{https://www-fourier.univ-grenoble-alpes.fr/\home{parisse}/casio/excas90.zip}{Graph 90} 
\item \ahref{https://www-fourier.univ-grenoble-alpes.fr/\home{parisse}/casio/excas35.zip}{Graph 35eii}
\end{itemize}
Sur la calculatrice, tapez MENU Système F6 F5 (OS update),
branchez la calculatrice à un PC sous Windows et
lancez le programme de mise en compatibilité inclus
\verb|tai-installer-*| (trusted addins installer)
et suivez les instructions sur la calculatrice. 

Vous pouvez maintenant installer KhiCAS. Branchez la calculatrice
sur un ordinateur, sélectionnez F1 sur la calculatrice, vous devez
voir les fichiers de la calculatrice vue comme une clef USB
sur votre gestionnaire de fichiers
\begin{itemize}
\item Graph 90e: copiez les fichiers {\tt khicasfr.g3a}, {\tt cas90a.mzs}
et {\tt cas90b.mzs} sur la calculatrice-clef USB
\item Graph 35eii: copiez le fichier {\tt excas.g1a} sur la
calculatrice-clef USB.
\end{itemize}

Il est conseillé de tester que tout s'est bien passé.
\begin{itemize}
\item Vérifiez que Khicas apparait dans le menu principal
Casio et se lance normalement. Tapez sur MENU pour revenir au menu principal
Casio.
\item \'Eteignez la calculatrice. Puis
appuyez sur les touches M et E
en même temps que la touche ON pour lancer le mode examen.
\item Vérifiez que vous pouvez lancer KhiCAS.
{\bf Attention, le lancement de KhiCAS en mode examen depuis le MENU 
principal Casio prend environ 6 secondes sur une Graph 90e et 
12 secondes sur une 35eii}.
Ceci vient de l'authentification cryptographique
de l'addin KhiCAS sur les deux modèles et des fichiers cas90*.mzs sur la 90. 
(Si vous appuyez sur la touche MENU depuis KhiCAS sans ouvrir une
autre application, vous pouvez réouvrir KhiCAS sans attendre ce
délai en appuyant à nouveau sur la touche MENU).
\item Quittez le mode examen en branchant la calculatrice sur un PC,
tapez F1 sur la calculatrice,
effacez un fichier inutile ou créez un fichier sur la calculatrice-clef USB.
\end{itemize}

Vous pouvez passer à la section \ref{sec:1erpas} Premier pas pour
apprendre à utiliser l'interface du shell, stocker des valeurs
dans des variables, saisir efficacement un nom de commande.
Si vous êtes en terminale, vous pouvez ensuite consulter
la section \ref{sec:bac25}
qui donne des indications sur l'utilisation de KhiCAS
pour vérifier ou effectuer les calculs d'un sujet de bac 2025. 

\subsection{Installation sur calculatrices (hors mode examen)} 
Pour installer ou mettre \`a jour $\chi$CAS
r\'ecup\'erez sur votre ordinateur 
\begin{itemize}
\item Sur les Graph 90, la version plus compl\`ete de $\chi$CAS
en trois fichiers~: le programme de lancement
\ahref{https://www-fourier.univ-grenoble-alpes.fr/\home{parisse}/casio/khicasfr.g3a}{khicasfr.g3a}
et les deux fichiers compressés
\ahref{https://www-fourier.univ-grenoble-alpes.fr/\home{parisse}/casio/cas90a.mzs}{cas90a.mzs}
\ahref{https://www-fourier.univ-grenoble-alpes.fr/\home{parisse}/casio/cas90b.mzs}{cas90b.mzs}
%% \\
%% ou si la version courte (sans g\'eom\'etrie, ...)
%% vous suffit, le fichier
%% \ahref{https://www-fourier.univ-grenoble-alpes.fr/\home{parisse}/casio/khicas.g3a}{khicas.g3a}
\item Sur Graph 35eii,
\ahref{https://www-fourier.univ-grenoble-alpes.fr/\home{parisse}/casio/excas.g1a}{excas.g1a}
\item Sur les Graph Math+, commencez par installer le
\ahref{https://www.planet-casio.com/storage/forums/mpm-installer-1.0bw-199064.zip}{MPM (par Lephe)}
(plus d'infos \ahref{https://www.planet-casio.com/Fr/forums/topic18534-1-mpm-mod-add-ins-math.html}{ici}). 
Puis récupérez les fichiers
\ahref{https://www-fourier.univ-grenoble-alpes.fr/\home{parisse}/casio/mpm/mpm.bin}{mpm.bin} (attention, il faut impérativement installer ce fichier
en remplacement de celui proposé par Lephe),
\ahref{https://www-fourier.univ-grenoble-alpes.fr/\home{parisse}/casio/mpm/khicasmp.g3a}{khicasmp.g3a},
\ahref{https://www-fourier.univ-grenoble-alpes.fr/\home{parisse}/casio/mpm/mpcasa.mzs}{mpcasa.mpz} et
\ahref{https://www-fourier.univ-grenoble-alpes.fr/\home{parisse}/casio/mpm/mpcasb.mzs}{mpcasb.mpz}.
\end{itemize}

Branchez le cable USB de la calculatrice, tapez F1 pour que la
calculatrice soit consid\'er\'ee comme une clef USB et copiez le ou
les fichiers 
\begin{itemize}
\item Graph90: \verb|khicasfr.g3a|, \verb|cas90a.mzs| 
et \verb|cas90b.mzs| 
\item Graph35eii: \verb|excas.g1a| 
\item Graph Math+: \verb|mpm.bin|, \verb|khicasmp.g3a|,
\verb|mpcasa.mpz| et \verb|mpcasb.mpz|
\end{itemize}
sur la ``clef''-calculatrice puis suivre
la manipulation de votre ordinateur qui permet de d\'ebrancher
une clef USB en toute s\'ecurit\'e. Le transfert s'effectue en 
moins d'une minute. 

%% {\bf Attention}, plusieurs utilisateurs ont fait part de
%% crashes d'addins avec la version 3.80.0 de l'OS Casio
%% (03.80.22{\bf 0}2 sur Graph 90+E, 03.80.02{\bf 0}2 sur fx-CG50, 
%% 03.80.12{\bf 0}2 sur fx-CG50AU).
%% Si vous avez des problèmes, essayez d'installer la version 3.80.1
%% \ahref{https://tiplanet.org/forum/archives_voir.php?id=4173827}{Windows},
%% \ahref{https://tiplanet.org/forum/archives_voir.php?id=3338352}{Mac}.
%% Si vous rencontrez toujours des problèmes,
%% il est possible de revenir à la
%% \ahref{https://tiplanet.org/forum/archives_list.php?cat=OS+cprizm}{version 3.70}.


\subsection{\'Emulateur}
Si vous testez sur l'\'emulateur
(\ahref{https://tiplanet.org/forum/archives_voir.php?id=1209556}{PC},
\ahref{https://tiplanet.org/forum/archives_voir.php?id=1352720}{Mac}), 
depuis le menu principal de la calculatrice (MENU), allez \`a M\'emoire,
puis F3 (Import/Export), puis F1 (Import files),
puis s\'electionnez le fichier \`a
transf\'erez depuis le disque dur de votre ordinateur, tapez
F1 pour sauvegarder sur la racine de la calculatrice, confirmez par F1 
si vous effectuez une mise \`a jour. Le transfert est relativement
long (plusieurs minutes) sur l'\'emulateur. 
Une fois le transfert termin\'e, une nouvelle icone (avec le flocon
de Xcas ou le mot Xcas) apparait dans le menu principal de la
calculatrice.

Attention, pour installer la version
compl\`ete en 2 fichiers sur l'\'emulateur, il faut transf\'erer
\ahref{https://www-fourier.univ-grenoble-alpes.fr/\home{parisse}/casio/khicas90.882}{khicas90.882}
et 
\ahref{https://www-fourier.univ-grenoble-alpes.fr/\home{parisse}/casio/emucas90.g3a}{emucas90.g3a}.

{\bf Remarques}:
\begin{itemize}
\item $\chi$CAS n'est pas compatible avec le simulateur distribu\'e
par Casio sur clef USB. Il faut bien installer l'\'emulateur, avec une
licence d'essai de 90 jours, les enseignants peuvent obtenir gratuitement
une licence, la demander \`a Casio Education France.
\item la version Windows de l'\'emulateur Casio tourne sous Linux avec
  wine, en utilisant la commande suivante apr\`es installation~:
\verb|wine "C:\Program Files (x86)\CASIO\fx-CG Manager PLUS Subscription for fx-CG50series\fx-CG_Manager_PLUS_Subscription_for_fx-CG50series.exe" /n"fx-CG Manager PLUS Subscription for fx-CG50series" &|
\end{itemize}

\section{Premiers pas} \label{sec:1erpas}
Depuis le menu principal, d\'eplacez le curseur jusqu'\`a l'icone de Xcas,
puis tapez EXE. Lors de la premi\`ere ex\'ecution, on vous demandera de choisir
entre syntaxe Xcas et Python. S\'electionnez l'un des deux avec
F1 et F6. Pour changer de syntaxe plus tard, utilisez shift-SETUP ou
le menu F6.

%% N.B.~: si vous utilisez la version longue de Xcas en 2 fichiers et que
%% le message ``Unable to load ram part'' apparait, il faut mettre \`a
%% jour l'OS de votre calculatrice (vers la version 3.30 ou plus).

Vous devez alors voir un historique de calculs (shell) vide 
dans lequel vous pouvez
taper la plupart des commandes de calcul formel de Xcas.

Par exemple, tapez 
\verb|1/2+1/6|
puis EXE, vous devriez voir le r\'esultat \verb|2/3| s'afficher
sur la ligne du dessous.

Vous pouvez recopier dans la ligne de commande
une commande de l'historique en utilisant le
curseur vers le haut ou vers le bas puis EXE, puis vous pouvez
modifier la commande et l'ex\'ecuter. Par exemple, taper sur la
touche curseur vers le haut, EXE et remplacez \verb|1/6| par \verb|1/3|.

{\bf Remarques~:}
\begin{itemize}
\item pour saisir un seul caract\`ere alphab\'etique en majuscule,
taper ALPHA en rouge puis la lettre. Pour saisir plusieurs caract\`eres
majuscules, taper SHIFT ALPHA. Pour saisir plusieurs caract\`eres en
minuscules, taper F5. F5 permet ensuite de basculer entre majuscules
et minuscules.
\item si vous ne voyez pas un caractère sur le clavier de
la calculatrice, vous pouvez sélectionner ce caractère depuis
la table de caractères en tapant les touches shift INS.
\item Lors de la premi\`ere utilisation, vous pouvez mettre \`a l'heure
l'horloge de la calculatrice en tapant la
commande \verb|hh,mm=>,|.
Pour saisir \verb|=>|, tapez sur la touche $\rightarrow$
située au-dessus de la touche AC/ON.\\
Par exemple \verb|13,17=>,| pour 13h17.
L'affichage de l'heure est utile en examen, car vous n'avez 
alors pas accès à un smartphone ou à une montre connectée.
\end{itemize}

Vous pouvez utiliser le r\'esultat de la derni\`ere commande
avec la touche \verb|Ans| de la calculatrice (taper sur shift 
puis \verb|(-)|),
mais il vaut en g\'en\'eral mieux d\'efinir une variable comme r\'esultat 
d'une commande si on souhaite la r\'eutiliser. Pour cela, on
utilise une des deux instructions d'affectation~:
\begin{itemize}
\item l'affectation vers la droite \verb|=>| s'obtient avec la 
touche $\rightarrow$ de la 
calculatricen (au-dessus de ON), 
par exemple \verb|2=>A| met 2 dans la variable A.
Vous pouvez ensuite utiliser \verb|A| dans un calcul, sa
valeur sera remplac\'ee par 2.
\item l'affectation vers la gauche se fait en tapant sur la touche
shift-$\rightarrow$.
Cela correspond \`a l'affectation en
Python (\verb|=|) ou en Xcas (\verb|:=|, \verb|=|
est accept\'e s'il n'y a pas de risque de confusion).
Par exemple \verb|A:=2| fait la m\^eme chose que \verb|2=>A|.
\end{itemize}
%Si vous commencez une ligne de commande en tapant la touche $\rightarrow$,
%le syst\`eme ins\'ere automatiquement \verb|Ans=>| ce qui permet 
%facilement de donner un nom au r\'esultat d'une commande.

Pour vous aider \`a saisir les commandes Xcas les plus utiles,
$\chi$CAS dispose d'un catalogue d'une centaine de commandes
par thèmes,
avec une courte description et le plus souvent un exemple 
d'ex\'ecution facile \`a recopier. 
Appuyez sur la touche F4 (cmds), choisissez une cat\'egorie avec le curseur, 
par exemple \verb|Algebre|, tapez EXE,
puis choisissez une commande avec le curseur, par exemple \verb|factor|.
La touche F6 vous affiche une courte description de la commande,
en g\'en\'eral avec un exemple. En tapant sur F2, vous recopiez
l'exemple en ligne de commande. S'il y a un deuxi\`eme exemple,
on le recopie avec F3. Vous pouvez alors valider (EXE) ou
modifier la commande et valider (EXE) 
pour factoriser un autre polyn\^ome que celui
donn\'e en exemple.

Si vous connaissez le nom de la commande, vous pouvez aussi
sélectionner la commande depuis l'index des commandes
en tapant sur les touches shift CATALOG (touche 4).

Lorsqu'une commande renvoie une expression (i.e. un r\'esultat qui
n'est pas entier, ni r\'eel approch\'e, ni une fraction d'entiers), 
celle-ci est affich\'ee
en \'ecriture naturelle (affichage 2-d). Vous pouvez faire d\'efiler 
l'affichage avec les touches du curseur lorsque l'expression est grande.
Tapez sur EXIT pour revenir au shell.

Maintenant essayez de taper la commande \verb|plot(sin(x))|.
Indication: taper F4 (cmds), puis s\'electionner 
\verb|Courbes|.

Lorsqu'une commande renvoie un graphe, celui-ci est affich\'e. Vous
pouvez modifier la fen\^etre graphique d'affichage 
avec les touches \verb|+| ou \verb|-| 
(zoom in ou out, utiliser la touche \verb|(-)|
pour faire un zoom out partiel selon l'axe $Oy$),
les touches du curseur, orthonormaliser le rep\`ere (touche \verb|/|)
ou faire une recherche automatique de l'\'echelle (autoscale 
touche \verb|*|). Pour enlever ou remettre les axes et graduations,
tapez sur SIN.
Tapez sur EXIT pour revenir au shell.

Vous pouvez effacer l'historique des calculs et les variables
pour commencer un nouvel exercice~: depuis le menu Fich,Cfg (F6) 
s\'electionnez \verb|Effacer l'historique|. 

Un certain nombre de raccourcis claviers sont pr\'evus pour faciliter 
les taches les plus courantes: par exemple F1 (alg) et F2 (calc) ouvrent un menu
avec des commandes d'alg\`ebre et d'analyse (calculus en anglais),
cf. la section \ref{sec:raccourcis} pour une liste compl\`ete. 
L'appui sur ON/AC efface la ligne courante, et si celle-ci
est vide propose d'effacer l'historique des calculs.

Pour quitter $\chi$CAS, appuyez sur la touche MENU. Lorsque vous lancez
une autre application, les variables et l'historique des calculs
sont sauvegard\'es dans le fichier {\tt session.xw}
(mémoire de sauvegarde de la calculatrice, visible
depuis l'application Mémoire de Casio), 
ils seront restaur\'es lorsque vous reviendrez
dans $\chi$CAS. 
La premi\`ere sauvegarde prend un peu de temps, en particulier 
sur l'\'emulateur,
ensuite les sauvegardes sont plus rapides. 
Vous pouvez sauvegarder plusieurs sessions distinctes (historique et
programmes \'eventuels) depuis le menu F6 de KhiCAS, et les
restaurer ult\'erieurement.

\section{Utilisation de la calculatrice et de KhiCAS au bac, exemples} \label{sec:bac25}
\subsection{Observations générales}
\subsubsection{100\% du sujet de Terminale.}
Sur les sujets de 2025, il y a toujours 4 exercices (4 à 6 points par exercice)
\begin{itemize}
\item un exercice de proba/stats. Pour cet exercice, une simple
calculatrice scientifique suffit, sauf en général pour une petite
question demandant un calcul de distribution binomiale cumulée qui
serait un peu fastidieux avec une calculatrice collège. Ce sera l'occasion
de tester le shell de KhiCAS sur des calculs simples. Les
commandes à connaitre sont {\tt binomial} et
{\tt binomial\_cdf} du menu 9. Probabilités.
\item presque toujours un exercice faisant intervenir une suite récurrente.
Pour cet exercice, on peut obtenir une représentation graphique
et la valeur numérique des premiers termes (N.B.: le calcul de 
quelques termes d'une suite
récurrente se fait déjà sur certains modèles college)
avec l'application Suites de la Casio, mais on ne peut pas faire
grand chose d'autre. 
L'équivalent dans KhiCAS est la commande
{\tt plotseq} (menu F3) qui permet de faire cela depuis le shell.\\
En exclusivité dans KhiCAS,
la commande {\tt rsolve} permet de déterminer l'expression
explicite d'une partie des suites récurrentes données (les
arithmético-géométriques, mais pas les logistiques). Dans
ce cas {\tt limit} permet de déterminer la limite de la suite.
Sinon, la commande {\tt solve} ou {\tt fsolve}
permet d'appliquer le théorème du point fixe pour trouver
la limite.
\item un exercice de géométrie analytique dans l'espace
souvent présenté sous la forme d'un vrai/faux. Les
applis intégrées Casio ne servent pas directement sur cet exercice.
Sur les Graph 90, KhiCAS permet de résoudre ou vérifier toutes les 
questions (exclusivité KhiCAS!),
et c'est probablement l'exercice où KhiCAS est vraiment utile,
car contrairement aux autres exercices, les réponses ne sont
pas données.\\
Les commandes sont dans le menu F4 Géométrie ou F4 3d ou directement
depuis shift CATALOG. Les noms de commande sont très simple à mémoriser,
il s'agit en général
d'un nom commun en français, par exemple {\tt point, droite,
plan, coordonnees, distance, equation, parameq}, etc.\\
Ces commandes ne sont hélas pas disponibles sur la 35eii par manque
de place.
\item Un exercice d'analyse, avec tableau de variations,
étude de convexité, calcul d'intégrale ou/et équation
différentielle. Le CAS de KhiCAS permet alors de vérifier tous les calculs
littéraux alors que les applis intégrées de Casio permettent
seulement de faire un graphe et des calculs numériques.
On utilise les commandes des menus F1, F2, F3 ainsi que F4 4 Analyse.
\end{itemize}
Grosso modo, on peut dire qu'une calculatrice scientifique couvre environ 
de 25\% à 40\% d'un sujet de bac
(40\% pour une scientifique avec distribution binomiale,
solveur numérique et calcul approché d'intégrale), 
une calculatrice graphique sans CAS couvre
environ 50\%, une calculatrice CAS du constructeur environ 75\%, une 
Graph 90 avec KhiCAS couvre 100\%.

Remarques~:
\begin{itemize}
\item
Attention, les rédacteurs de sujet de bac utilisent souvent des
``nombres à virgule'' pour représenter des quantités numériques,
par exemple 0,6, sans indiquer s'il s'agit de valeurs exactes ou
d'estimations approchées. 
Dans KhiCAS, on distingue deux représentations
pour les nombres, la représentation approchée qui ici serait
{\tt 0.6} (avec un point séparant la partie entière de la partie
décimale) et la représentation exacte qui ici serait
{\tt 3/5}. Il est presque toujours indispensable d'utiliser la représentation
exacte pour faire du calcul littéral avec KhiCAS, et ne faire un
calcul approché qu'à la fin si c'est demandé. Il y a quelques
exceptions, par exemple si on souhaite calculer une 
solution approchée d'une équation (lorsqu'il n'y a pas
de méthode exacte pour le faire) ou calculer la valeur
approchée d'une intégrale 
(s'il n'existe pas de primitive explicite de la fonction).
\item Si vous êtes pris par le temps en examen ou si le résultat que
vous obtenez n'est pas cohérent, je vous conseille de l'indiquer 
sur votre copie, indiquez ensuite
la commande KhiCAS que vous avez utilisée sur la calculatrice sur votre 
copie ainsi que la réponse obtenue.\\
Si vous avez montré votre capacité à faire des calculs 
à la main ailleurs sur votre copie, vous pouvez espérer récupérer 
une partie des points de la question, à moins que le correcteur 
ne soit obtus.
\end{itemize}

\subsubsection{\`A propos de l'algorithmique au bac}
Il y a au plus une question dans le sujet (sur les sujets de métropole
dans un sujet sur les deux en 2024 et 2025). Elle est posée 
dans l'exercice sur les suites (je n'ai pas vu d'exception pour 2025, 
c'est en général une question stéréotypée sur un seuil). L'énoncé donne 
un script Python, soit avec des {\tt ...} à compléter (expliciter
le test d'arrêt ou/et traduire la récurrence en une instruction),
soit sous forme d'une fonction {\tt mystere} à interpréter, c'est
essentiellement évident dans le contexte de l'exercice. Il est
parfois demandé de donner le résultat d'exécution du script,
impliquant les premiers termes de la récurrence \\
L'utilisation de l'application Python de la calculatrice
n'a guère d'intérêt en pratique, il est nettement plus efficace d'utiliser
l'application Suites de la calculatrice ou la commande {\tt plotseq}
de KhiCAS (et regarder une dizaine de termes). Dans quelques cas
(exécution de script), on peut effectivement saisir le script
et l'exécuter, mais d'un point de vue compréhension, cela revient
à donner des points à un élève qui est un bon presse-bouton au lieu
de donner des points à un élève qui a compris l'algorithme!\\
Dans d'autres cas, exécuter le script risque fort de dérouter les
candidats, parce que
les concepteurs de sujet ne semblent pas tester systématiquement
les scripts donnés. Par exemple le script donné dans le sujet
de \ahref{https://www.apmep.fr/IMG/pdf/Polynesie_spe_J1_17_06_2025_DV.pdf}{Polynésie du 17 juin 2025}, est le suivant
\begin{giaconload}
from math import *
def mystere(n):
    I = 1 - 1.0/e
    L = [I]
    for i in range(n):
        I = (i+1)*I - 1/e
        L.append(I)
    return L
\end{giaconload}
et on demande ce que renvoie \\
\giacinput{mystere(100)}\\
Cela correspond aux 101 premiers termes (d'indice 0 à 100) de la suite
récurrente étudiée dans l'exercice
$$ I_n=\int_0^1 x^n e^{-x} \ dx, \quad I_{n+1}=(n+1)I_n-1/e $$
Or ce script est numériquement instable, et ne donne pas du tout les 
valeurs attendues. Les éléments de la liste $L$ deviennent négatifs au
bout d'une vingtaine de termes, et le dernier terme vaut 
{\tt -3.1530126806564304e+141} (avec CPython), laissant croire
que $\lim_{n \rightarrow +\infty}I_n=-\infty$ alors que la limite est nulle.
Cela s'explique par le fait qu'à chaque itération on soustrait des nombres
qui devraient être de plus en plus proche, ce qui est la principale
source de perte de précision dans un calcul numérique. Comprendre
cela est hors de portée d'un lycéen normal, surtout en examen, j'en
veux pour preuve que cela
a échappé aux concepteurs de sujet, aux testeurs, et la première
version du corrigé du site de l'APMEP n'en parle pas non plus.

Conclusion~: pendant l'examen, je vous déconseille d'utiliser
l'appli Python du constructeur de votre calculatrice ou celle 
incluse dans KhiCAS. Les questions sont suffisamment simples
pour être résolues papier-crayon ou plus efficacement
avec l'application Suites de la calculatrice.
Et du coup, on se demande à quoi cela rime d'avoir imposé Python
comme unique langage au lycée, en tout cas pour les sujets de bac.
il serait bien plus raisonnable à mon avis d'écrire des algorithmes 
en langage naturel puisque visiblement on se dispense de l'étape de test 
des scripts, seul intérêt de fixer un langage, surtout à l'heure où
l'IA générative n'a aucun mal à traduire vers un langage donné.
On éviterait en tout cas les problèmes
liés à la représentation approchée des nombres.

\subsubsection{L'épreuve anticipée de 1ère et le DPE calculatrice}
En juin 2025, l'épreuve anticipée de math de 1ère a été confirmée
avec deux parties (automatismes et résolution de problèmes).
La calculatrice est pour le moment interdite pendant toute
l'épreuve. C'est à mon sens ridicule d'interdire la calculatrice
dans la 2ème partie, résolution de problèmes, alors que c'est
légitime pour juger les automatismes. C'est aussi ce que
demandait l'APMEP, mais nos décideurs ont choisi de passer outre,
suivant les demandes des intégristes anti-calculatrice.
On voit ainsi sur les sujets 0 un encadré ``Aide au calcul''
qui peut par exemple souffler aux candidats que 0,5*0,5=0,25 (sic). 
On n'est pas loin
du retour à des tables d'exponentielles (le log n'est pas au programme
de 1ère) et trigonométriques.

Le coefficient de cette épreuve est faible pour le bac (2/100), 
mais il s'agit de la seule épreuve de sciences hors controle continu
qui figurera dans le dossier des candidats à Parcoursup, elle aura
donc un impact bien au-delà de ce coefficient. Il est donc
probable qu'on assiste à un recul massif de l'utilisation un peu évoluée
de calculatrices graphiques (et de logiciels de calcul) au lycée, 
au moins en première (en vue de la préparation à l'épreuve
anticipée), mais aussi
en seconde~: les élèves de seconde ne vont probablement plus s'équiper,
ce sera reporté à la fin de 1ère début terminale
pour les élèves ayant choisi la spécialité maths en terminale
(175 000 en 2025). Une partie de ces élèves
va sans doute trouver l'achat
d'une calculatrice graphique milieu de gamme
(TI83, Casio Graph90/math+, Numworks) trop onéreux 
juste pour quelques mois (septembre à
mars, date de fin des dépôts de dossier Parcoursup), ils vont utiliser
une calculatrice scientifique (qui permet de traiter de 20\% à
40\% d'un sujet de bac selon le modèle). 
On peut anticiper une division par un facteur 2
de la taille du marché français des calculatrices graphiques,
peut-être plus si un cercle vicieux se met en place (hausse
du prix de vente car moins d'achats, pressions pour que les sujets de 
terminale soient faisable avec une calculatrice collège). 

L'élève
qui achètera une calculatrice le fera ssouvent deux ans après son entrée
au lycée,
il sera plus autonome dans le choix  entre calculatrices
selon les capacités utiles pour la terminale et éventuellement la poursuite
d'études dans le supérieur. Pour aider à faire ce choix,
je propose d'établir un 
{\bf DPE, diagnostic perfomance examen}, de quelques calculatrices
en fonction des sujets de bac de spé maths terminale de 2025.
Un + désigne une fonctionnalité présente, un - présente mais peu confortable.
\begin{center}
\begin{tabular}{|c|c|c|c|c|c|c|c|c|c|c|} \hline 
Calculatrice   &{\bf Graph 90}& Graph Math+&{\bf Graph 35eii}& Graph Light & FX92 & Numworks 110 & Numworks 115/120 & TI83CE &TI82AEP& {\bf TI36XPRO} \\ 
observation    & KhiCAS   &             & KhiCAS      &             &      & déverrouillée&verrouillée       & KhiCAS & \\
prix rentrée   & occas ($\approx$60?) & 70-90    & {\bf 50}          &   35        & 15-20 & occas ($\approx$60?)  &  83            &  80-90  & 60  & {\bf 20} \\
DPE            &  {\bf A+}  &  C          &  {\bf B}        &   D         & E    & A            &    C             & B      &  C   &  D \\
proba-stats    & +        & +           &  +          &  +          &  +   & +            &    +             &  +     & +    &   + \\ 
binomial\_cdf  & +        & -           &  +          &   +         &  +   &  +           &    +             &  +     & +    &   + \\ 
graphe suite   &  +       & +           &  +          &             &     &   +          &     +            &   +    &  +   &    \\ 
premiers termes&  +       & +           &  +          &   -         &  -   & +            &    +             &  +     & +    &   - \\ 
rsolve         &   +      &             &   +         &             &      &  +           &                  &        &      &    \\ 
limit          &   +      &             &   +         &             &      &  +           &                  &  +     &      &    \\ 
solve          & +        &             &  +          &             &      &  +           &                  &  +     &      &    \\ 
fsolve         &  +       &  +          &  +          &   +         &      &  +           &    +             &  +     &  +   &   + \\ 
analytique 3D &   +       &             &             &             &      &  +           &                  &        &      &    \\ 
tests 3D      &   +       &             &             &             &      &  +           &                  &        &     &    \\ 
géométrie 3D  &   +       &             &             &             &      &  +           &                  &        &     &    \\ 
linsolve      &   +       &  +          & +           &  +          &   -  &  +           &   +              &  +     &   +  &   + \\ 
graphe fct    &   +       &  +          &  +          &   +         &      &  +           &   +              &  +     &   +  &    \\ 
dérivée       &   +       &             &   +         &             &      &  +           &                  &  +     &      &    \\ 
primitive     &   +       &             &   +         &             &      &  +           &                  &  +     &      &    \\ 
aire/courbe   &   +      &  +           &  +          &   +         &      &  +           &   +              &  +     &      &    + \\ 
tableau var.  &  +        &             &   +         &             &      &  -           &                  &  +     &      &    \\ 
mémoire RAM   &  8M      & 8M         & 512K        &  24K          &24K   & 256K         &  256K           & 256K    & 256K & N.A. \\
stockage flash&  13M/32  & 4.7M/32    & 1M/8        &  0/512K       &0/512K& 2M/8        & 5M/8             & 192K/4M & /4M  & 0/128K \\
\end{tabular}
\end{center}
Voir aussi \ahref{https://tiplanet.org/forum/compare.php}{le QCC de tiplanet}
qui établit une comparaison beaucoup plus complète, mais celui-ci
est adapté au programme de spé maths de terminale. Pour les prix
de calculatrices neuves, cf. 
\ahref{https://tiplanet.org/prix/}{le comparateur de tiplanet}.

Remarques~:
\begin{itemize}
\item Les deux calculatrices classées A sont donc des calculatrices qui ne
sont plus produites, à acheter principalement d'occasion. Attention,
pour la Numworks, bien demander si elle est déverrouillée, pas
de risques de ce type avec la Graph 90.
La Graph 90 est de toutes façons un meilleur choix à mon avis, 
le clavier a plus de touches,
il y a beaucoup plus de mémoire RAM disponible (peu de risque
que KhiCAS doive interrompre un calcul par manque de mémoire), il
y a beaucoup plus d'espace de stockage disponible, et il est très facile
d'y accéder (la calculatrice se comporte comme une clef USB).
La Numworks a un processeur plus rapide.
\item La Graph Math+ ressemble beaucoup à la Graph 90, et pourrait passer
de C à A si on rend KhiCAS compatible en mode examen. De même pour les
Numworks N0115/N0120.
\item Si on se tourne vers une calculatrice scientifique 
pour des raisons économiques, la fx92 collège,
qui est le modèle le plus populaire au collège en France, 
me parait quand même trop limitée,
sauf si Casio décidait d'y ajouter les fonctionnalités qu'on retrouve
dans les calculatrices scientifiques non destinées aux scolaires. La Graph
Light est vendue bien trop chère, alors que son coût de fabrication
matériel est certainement très proche de la fx92, on devrait 
la trouver vers les 25 euros, et pas 35! Pour
un prix essentiellement équivalent à la fx92 (soit 20 euros), je conseillerais
plutôt la TI36XPRO, qui est moins profilée scolaire, mais permet
de traiter 40\% d'un sujet de bac de terminale spé maths (et dispose
en plus d'un peu de calcul matriciel pour maths expertes), 
on est très proche (le confort en moins) d'une
Numworks verrouillée ou d'une Graph Math+ pour 4 fois moins cher
(et même 5 fois si on l'achète dès le collège!).
\end{itemize}
{\bf Conclusion}~: selon votre budget, je vous conseille la Graph 90 
graphique couleur (à trouver d'occasion vers les 60 euros, 
les prix neufs sont en général bien trop élevés), 
ou la Graph 35eii graphique monochrome 
(neuf 50 euros avec la remise différée de 14 euros de Casio) ou
ou la TI36XPRO scientifique monochrome (neuf 20 euros).

\subsection{Premier exemple de sujet}
Le sujet du 17 juin
\ahref{https://www.apmep.fr/IMG/pdf/Metro_J1_17_06_2025_DV.pdf}{métropole 2025 jour 1}

\subsubsection{Exercice 1}
Tout le début peut se faire avec une calculatrice scientifique, 
on peut bien sur aussi utiliser le shell de KhiCAS.

Question 1.b, on gagne un peu de temps avec\\
\giacinput{binomial_cdf(100,0.0714,7)}\\
qui renvoie 0.577. Pour saisir la commande, le plus efficace est
de taper shift CATALOG puis b i n, descendre jusqu'à 
\verb|binomial_cdf|. F5 affiche alors l'aide de la commande, puis
on tape F1 ou F2 pour recopier un exemple, on modifie et on tape EXE.
On peut vérifier que c'est vraissemblable avec le graphe\\
\giacinput{plot(binomial(100,0.0714))}

1.c
\giacinput{stddev(binomial,100,0.0714)=>S}\\
permet de calculer l'écart-type de la loi qui vaut 2.57.. et
de la stocker dans {\tt S}.
Ensuite \verb|S^2| donne la variance 6.63...

2.d
\giacinput{S/sqrt(N) => s}
nous donne l'écart-type de $M_N$ qui 
va servir ensuite pour Bienaymé-Tchebycheff. 
On veut $P(|X-7.14|>=0.14)<0.05$, on pose donc $0.14=a*s$, 
avec $a$ tel que \\
\giacinput{solve(1/a^2=0.05,a) => A}, 
puis $N$ tel que \\
\giacinput{solve(A[1]*s=0.14,N)} qui renvoie N=6766.

Discussion : cela fait beaucoup de villes, mais on surestime avec l'inégalité de Bienaymé-Tchebycheff, en utilisant la loi normale $a$ passe de 4.47 à 1.96, 
une estimation plus réaliste diviserait donc $N$ par 5 environ, 
ce qui correspond au nombre de villes de plus de 5000 habitants 
(ça reste 10 fois plus que le nombre de maisons du don, mais beaucoup 
moins que le nombre de collectes mobiles). 

\subsubsection{Exercice 2}
{\bf Partie A}~:\\
Cette partie est prévue pour tester la lecture graphique 
et ne pas utiliser d'outil de calcul,
mais comme la fonction est donnée en partie B, on peut quand même 
vérifier (trouver?) les valeurs de la partie A avec KhiCAS.
On commence par saisir la valeur de f donnée en partie B\\
\giacinput{x*(2*ln(x)^2-3*ln(x)+2) => F}\\
puis on vérifie qu'elle correspond au graphe de la partie A\\
\giacinput{plot(F,x=0..3)}

Pour ajouter la tangente en $x=1$, il faut donner un nom au graphe, 
on remonte dans l'historique pour recopier la commande précédente et on ajoute
\verb|G:=| (taper shift $\rightarrow$ pour saisir \verb|:=|)\\
\giacinput{G:=plot(F,x=0..3)}
puis on tape \\
\giacinput{TA:=tangent(G,1)}
et pour voir $G$ et $TA$ simultanément\\
\giacinput{G,TA}

La commande\\
\giacinput{equation(TA)}
nous renvoie $y=-x+3$ et permet de vérifier qu'on passe par le point $C$, 
le coefficient directeur -1 nous donne la valeur de $f'(1)$, qu'on
peut ausi obtenir par les deux commandes\\
\giacinput{diff(F)=>F1} 
(taper F2 pour saisir \verb|diff| ou \verb|'|)\\
\giacinput{F1(x=1)}\\
On vérifie que \\
\giacinput{solve(F1=0)} \\
renvoie 2 solutions ($1/e$ et $e^{1/2}$).
On calcule\\
\giacinput{F1'=>F2}\\
puis\\
\giacinput{F2(x=0.2)}

{\bf Partie B}~:\\
Utiliser les menus F1 et F2 pour saisir rapidement les commandes\\
\giacinput{solve(2X^2-3X+2,X)}\\
\giacinput{limit(F,x,oo)}\\
\giacinput{limit(F,x,0)} (admis dans l'énoncé)\\
\giacinput{simplify(F2)}\\
\giacinput{solve(F2>0)}

{\bf Partie C:}\\
\giacinput{equation(tangente(G,e))}
\giacinput{integrate(x*ln(x),x)} (pour vérifier la primitive)\\
\giacinput{integrate(x*ln(x),x,1,e)} (pour vérifier l'énoncé)\\
\giacinput{integrate(F-(2x-e),x,1,e)}
renvoie $3*exp(1)^2/4-(4*exp(1)+5)/4$\\
En mettant 1.0 au lieu de 1 on a une valeur numérique 1.57... 
et on peut juger si c'est vraisemblable en comparant à l'aire 
de la partie hachurée sur l'énoncé.

\subsubsection{Exercice 3}
Pour saisir les commandes, taper les touches F4 * (3d) ou F4 $\rightarrow$
(géométrie) ou shift CATALOG et les premières lettres de la commande\\
\giacinput{point(-1,0,5)=>A}

\giacinput{point(3,2,-1)=>B}

\giacinput{point(0,0,0)=>O}

Sur la calculatrice taper EXIT pour quitter le graphe 3d.
Pour visualiser plusieurs objets géométriques simultanément, on peut taper ici
\giacinput{A,B,O}. Les touches du curseur permettent de changer le point de vue
3d.

Affirmation 1:VRAI \\
\giacinput{B-A} renvoie les coordonnées du vecteur 
$\overrightarrow{AB}$ et on regarde si c'est colinéaire avec 
le vecteur directeur déduit de l'équation paramétrique, 
qui passe en $t=0$ par $B$. 
Attention une droite admet une infinité de représentations paramétriques, 
{\tt parameq(droite(A,B))} 
ne donne pas la même équation que celle de l'énoncé.

Affirmation 2: FAUX\\
\giacinput{equation(plan(O,A,B))} 
renvoie $-10*x+14*y-2*z=0$ donc vecteur normal au plan $(-10,14,-2)$ 
non colinéaire au vecteur $n$. 
Autre solution \giacinput{cross(A-O,B-O)}

Affirmation 3: VRAI\\
On voit sans calcul que les 2 droites ne sont pas parallèles 
(les vecteurs directeurs ne sont pas colinéaires),
on teste si elles sont sécantes
\giacinput{solve([15+k=1+4s,8-k=2+4s,-6+2k=1-6s],[k,s])} 
renvoie [[-4,5/2]] donc oui, donc les droites sont coplanaires.

Affirmation 4: VRAI\\
\giacinput{point(2,-1,2)=>C}

\giacinput{distance(plan(x-y+z+1=0),C)} renvoie $2\sqrt{3}$.
On peut aussi le faire de tête en remplaçant les coordonnées de $C$ 
dans l'équation du plan (ce qui donne 6) et en divisant par la racine 
carrée de la somme des carrés des coefficients de x,y,z 
(norme du vecteur normal, $\sqrt{3}$)

\subsubsection{Exercice 4}
{\bf Partie A}\\
\giacinput{-0.02*x^2+1.3*x=>H}\\
\giacinput{H(x=1)} nous donne u1=1.28
\giacinput{solve(H=x)} nous donne la limite $L=15$ ($L=0$ est exclus ici)

{\bf Partie B}\\
La résolution de la question 1 à la machine est un peu technique, 
je pense qu'il vaut mieux le faire sans la calculatrice 
d'autant plus que la réponse est donnée. Si on veut vraiment le vérifier,\\
\giacinput{eq:=diff(y(t),t)=0.02*y(t)*(15-y(t))}\\
\giacinput{normal(equal2diff(subst(eq,y(t)=1/f(t))))}\\
qui affiche \verb|-diff(f(t),t)-0.3*f(t)+0.02|\\
Pour vérifier la solution de l'équation différentielle, il faut
saisir les coefficients sous forme de fraction
\begin{itemize}
\item sans condition initiale \\
\giacinput{desolve(y'=-3/10*y+2/100)} renvoie \verb|(15*c_0*exp(-3*x/10)+1)/15|
\item
avec condition initiale\\
\giacinput{desolve([y'=-3/10*y+2/100,y(0)=1])=>g} 
renvoie \verb|(14*exp(-3*x/10)+1)/15|
\end{itemize}
Puis on inverse 
\giacinput{1/g=>f}\\
puis on vérifie 
\giacinput{limit(f,x,oo)} 
on retrouve le $L$ de la partie A. Enfin
\giacinput{solve(f>14)}  renvoie $x>10*\ln(196)/3$


\subsection{Deuxième exemple de sujet}
Le sujet du 18 juin
\ahref{https://www.apmep.fr/IMG/pdf/Metropole_spe_J2_18_06_2025_DV.pdf}{métropole 2025 jour 2}

\subsubsection{Exercice 1}
Exercice de proba/stats: une calculatrice scientifique suffit ici, 
sauf pour A5b/ où ce serait fastidieux, dans KhiCAS on peut taper\\
\giacinput{binomial_cdf(100,0.24,20,100)} 
ou son équivalent à 3 arguments \\
\giacinput{1-binomial_cdf(100,0.24,19)}\\
(pour saisir {\tt binomial\_cdf}, taper shift CATALOG puis b i n
puis déplacer le curseur sur la commande, taper F5 si vous voulez
consulter l'aide en ligne, puis F1 ou F2 pour recopier un exemple,
le modifier et calculer avec EXE).

\subsubsection{Exercice 2}
{\bf Partie A}\\
Question 1/ sélectionner solve dans le menu rapide F2, puis saisir\\
\giacinput{solve([3/2+2t=s,2+t=3/2+s,3-t=3-2s],[t,s])}
(pour taper {\tt t} ou {\tt s}, taper F5 puis la touche avec
le label en rouge correspondant puis ALPHA pour quitter le mode alphabétique).
On obtient {\tt [[-1,-1/2]]} donc $t=-1$ (et $s=-1/2$), 
les droites sont bien sécantes, pour déterminer l'intersection on peut
faire\\
\giacinput{subst([3/2+2t,2+t,3-t],t=-1)=>S} ({\tt =>} s'obtient
en tapant sur la touche $\rightarrow$ située au-dessus de AC/ON).
Attention {\tt S} n'est pas un point géométrique 
mais la liste des coordonnées de l'intersection.

Question 2/
On crée les points avec la commande {\tt point} du menu F4 * (3d)
ou shift CATALOG taper le début de point et déplacer le curseur
à la commande point.\\
\giacinput{point(-1,2,1)=>A} (taper ALPHA puis la touche de label A pour saisir A)

\giacinput{point(1,-1,2) => B}

\giacinput{point(1,1,1) => C}

\giacinput{[1,2,4] => n}\\
$A, B, C$ sont des objets géométriques 3d, qui sont représentés graphiquement
lorsqu'on exécute la commande. Taper EXIT pour quitter la représentation
graphique. Si vous voulez voir les trois points simultanément, taper
dans une ligne de commande \\
\giacinput{A,B,C}

Vous pouvez utiliser les touches 
curseur pour changer de point de vue 3d, EXIT pour quitter.\\
\giacinput{(B-A)*n} renvoie 0. La différence de 2 points est la liste
des coordonnées du vecteur les reliant ($\overrightarrow{AB}$ ici), 
et {\tt *} entre 2 vecteurs calcule leur produit scalaire.
\giacinput{(C-A)*n} renvoie aussi 0.
Donc $n$ est normal au plan.\\
\giacinput{subst(x+2y+4z,[x,y,z],coordonnees(A))} renvoie 7, 
c'est la bonne constante.

Question 3/ \\
\giacinput{subst(x+2y+4z,[x,y,z],S)} renvoie 35/2 donc le point $S$ dont les
coordonnées sont dans la variable {\tt S} n'est pas dans le plan $ABC$.

Question 4a/ \\
\giacinput{[-1,0,2]=>H}\\
\giacinput{H-S} renvoie [-1/2,-1,-2] soit $-n/2$ et\\
\giacinput{subst(x+2y+4z,[x,y,z],H)} renvoie 7.

Question 4b/ 
\giacinput{distance(H,S)} renvoie $\sqrt{21}/2$.

{\bf Partie B}\\
\giacinput{coordonnees(C)=>c}\\
\giacinput{c+k*(S-c)=>M} (représentation paramétrique des coordonnées de $M$)\\
\giacinput{(M-coordonnees(A))*(M-coordonnees(B))=>eq} (produit scalaire
$\overrightarrow{AM}.\overrightarrow{BM}$)\\
\giacinput{solve(eq=0,k)} renvoie {\tt list[(-6*sqrt(14)+12)/45,(6*sqrt(14)+12)/45]}, ou numériquement\\
\giacinput{fsolve(eq=0,k)}, il y a une seule solution sur le segment 
($k=0.76...$)

\subsubsection{Exercice 3}
1/ faux (Evident!) pour vérifier \\
\giacinput{limit((1+5^n)/(1+3^n),n,oo)} renvoie $+\infty$.

2/ N.B.: $w_0==0$ donc faux sur l'énoncé de l'apmep mis en lien, 
mais sur le vrai sujet on a inégalité large, donc c'est vrai.
Si on n'aime pass trop les inéquations, on peut montrer par récurrence que 
$w_n=n+3^n-1$ et conclure puisque $3^n-1>=0$.
En effet, KhiCAS permet de calculer l'expression explicite de $w_n$ \\
\giacinput{rsolve(w(n+1)=3*w(n)-2n+3,w(n),w(0)=0)} (rsolve est dans
le menu F2) qui renvoie $[n+3^n-1]$. Il suffit ensuite de le justifier.

3/ c'est concave en $A$

4/ \giacinput{tabvar(ln(x)-x+1)} (tabvar est dans le menu F4 Analyse,
ou bien shift CATALOG taper le premières lettres de la commande).
N.B. l'inégalité est large sur le ``vrai'' sujet. Donc vrai.

\subsubsection{Exercice 4}
{\bf Partie A lecture graphique}\\
Cette partie d'énoncé avec un graphique sur papier millimétré qui fleure
bon le 20ème siècle est prévu pour faire des lectures graphiques,
sans calculatrices. Mais en exclusivité ici, comme on ne demande pas 
de justifier dans cette partie, je vous explique comment le résoudre 
avec des outils du 21ème siècle. \\
Il suffit en effet de lire l'énoncé jusqu'au bout pour obtenir 
l'expression de $d$:\\
\giacinput{(12+x)*exp(-3/5*x) =>v}\\
\giacinput{integrate(v,x,0,t) => d} (menu F2)\\
\giacinput{plot(d,t=0..15)} (menu F3) permet de reproduire la figure pour
vérifier la saisie.
\giacinput{fsolve(d=15,t=2)} renvoie 2.02..., en lecture graphique 2
\giacinput{limit(d,t,oo)} renvoie 205/9 soit environ 22.8
\giacinput{v(x=4.7)} vaut 0.99... donc en lecture graphique 1.

{\bf Partie B}\\
1a/ 
\giacinput{desolve(y'+3/5*y=0)} ou \giacinput{desolve(y'+0.6*y=0)} (pour les 
étourdis qui ont oublié que ça se résoud de tête, desolve est dans le 
menu rapide F2).

1b/ \giacinput{desolve(y'+3/5*y=exp(-3/5*t))}

1c/ \giacinput{desolve([y'+3/5*y=exp(-3/5*t),y(0)=12])} 
(ou \giacinput{desolve(y'+3/5*y=exp(-3/5*t) and y(0)=12)})
renvoie {\tt t*exp(-3*t/5)+12*exp(-3*t/5)}\\
Ou en approché\\
\giacinput{desolve([y'+0.6*y=exp(-0.6*t),y(0)=12])}
renvoie t*exp(-0.6*t)+12.0*exp(-0.6*t)

1d/ Pour trouver $v$ sans se fatiguer à dériver, 
on peut utiliser l'équation différentielle

2/ 
\giacinput{v'=>v1}\\
\giacinput{normal(v1)} renvoie le résultat de l'énoncé en exact\\
\giacinput{limit(v,x,oo)} renvoie 0, on s'en doutait puisque $d$ converge (graphiquement ou vers 205/9).\\
\giacinput{tabvar(v,x=0..oo)} (menu F4 Analyse)\\
\giacinput{fsolve(v=1,x=4.7)=>T} permet de vérifier le résultat de la partie A (à recopier en 2d et 3)

{\bf partie C}: les calculs ont déjà été faits:\\
\giacinput{normal(d)}
\giacinput{d(t=T)} renvoie 20.94...
on vérifie sur le graphique l'ordonnée du point $A$.

Remarque~: ce modèle me semble partiellement sorti du chapeau, en effet 
si la partie homogène de l'équation différentielle sur $v$ 
correspond au principe fondemental de la dynamique avec
un frottement proportionnel à la vitesse, je ne vois
pas ce que modélise le terme de forçage
non homogène en $e^{-0.6t}$, dépendant du temps compté à partir du moment
où le chariot entre dans la zone de freinage. 
Est-ce un habillage artificiel, le but étant d'avoir une solution particulière
multipliée par $t$ (-0.6 solution de l'équation caractéristique)~?


\section{Commandes usuelles de calcul formel}
\subsection{D\'evelopper et factoriser}
Depuis le catalogue, s\'electionner le sous-menu \verb|Algebre|
ou tapez F1~:
\begin{itemize}
\item \verb|factor|~: factorisation. Raccourci clavier \verb|=>*|
(touche $\rightarrow$ puis *),
par exemple \giacinputmath{x^4-1=>*}.
Utiliser \verb|cfactor| pour factoriser sur $\C$.
\item \verb|partfrac|~: d\'eveloppement d'un polyn\^ome ou
d\'ecomposition en \'el\'ements simples pour une fraction. Raccourci
clavier \verb|=>+| (touche $\rightarrow$ puis +), 
par exemple \giacinputmath{ (x+1)^4=>+} 
ou \giacinputmath{1/(x^4-1)=>+}.
\item \verb|simplify|~: essai de simplifier une expression.
Raccourci clavier \verb|=>/| (touche $\rightarrow$ puis /), 
par exemple \giacinputmath{sin(3x)/sin(x)=>/}
\item \verb|ratnormal|~: d\'evelopper une expression,
\'ecrire une fraction sous forme irr\'eductible.
\end{itemize}

\subsection{Analyse}
Depuis le catalogue, s\'electionner le sous-menu \verb|Analyse|, ou
tapez F2
\begin{itemize}
\item \verb|diff|~: d\'erivation. On peut aussi utiliser la notation
\verb|'| (raccourci clavier F3) pour d\'eriver par rapport \`a $x$, ainsi
\giacinputmath{diff(sin(x),x)} et \giacinputmath{sin(x)'} sont \'equivalents.
Pour d\'eriver plusieurs fois, ajouter le nombre de d\'erivations
par exemple \giacinputmath{diff(sin(x^2),x,3)}.
\item \verb|integrate|~: primitive si 1 ou 2 arguments,
par exemple 
\giacinputmath{integrate(sin(x))} ou \giacinputmath{integrate(1/(t^4-1),t)}
pour $\int \frac{1}{t^4-1} \ dt $\\
Calcul d'int\'egrale d\'efinie si 4 arguments, par
exemple \giacinputmath{integrate(sin(x)^4,x,0,pi)}
pour $\int_0^\pi \sin(x)^4 \ dx$. Mettre une des bornes
d'int\'egration sous forme approch\'ee si on souhaite un calcul
approch\'e d'int\'egrale d\'efinie, par exemple
\giacinputmath{integrate(sin(x)^4,x,0.0,pi)}\\
Raccourci clavier shift F3.
\item \verb|limit|~: limite d'une expression.
Exemple \giacinputmath{limit((cos(x)-1)/x^2,x=0)}\\
Pour une limite \`a droite [resp. \`a gauche], ajouter 1 [resp. -1]
en dernier argument, par exemple
\giacinputmath{limit(1/x,x=0,1)}
\item \verb|tabvar|~: tableau de variations d'une expression.
Par exemple \giacinputmath{tabvar(x^3-7x+5)}
on peut v\'erifier avec le graphe \giacinput{plot(x^3-7x+5,x,-4,4)}
\item \verb|taylor| et \verb|series|~: d\'eveloppement de Taylor
(ou d\'eveloppement limit\'e ou asymptotique). Par exemple
\giacinput{taylor(sin(x),x=0,5)}
\item \verb|sum|~: somme discr\`ete. Par exemple
\giacinputmath{sum(k^2,k,1,n)} calcule $\sum_{k=1}^n k^2$,
\giacinputmath{sum(k^2,k,1,n)=>*} calcule la somme et l'\'ecrit sous
forme factoris\'ee.\\
Raccourci clavier ALPHA F3.
\end{itemize}

\subsection{R\'esoudre}
Depuis le catalogue, s\'electionner le sous-menu \verb|Resoudre|.
\begin{itemize}
\item \verb|solve| permet de r\'esoudre de mani\`ere exacte
une \'equation (se ramenant \`a une \'equation polynomiale).
Il faut pr\'eciser la variable si ce n'est pas \verb|x| par
exemple \giacinputmath{solve(t^2-1=0,t)}. Si la
variable est \verb|x|, on peut \'ecrire l'expression
ou l'\'equation suivie de 2 appuis sur la touche $\rightarrow$,
par exemple \verb|x^2-1=>|\\
Si la recheche exacte \'echoue, la commande \verb|fsolve|
permet de faire une r\'esolution approch\'ee, soit par
une m\'ethode it\'erative en partant d'une valeur
initiale \giacinputmath{fsolve(cos(x)=x,x=0.0)}, soit par
dichotomie  \giacinputmath{fsolve(cos(x)=x,x=0..1)}.\\
Pour avoir des solutions complexes, utiliser \verb|csolve|.\\
On peut faire des hypoth\`eses sur la variable que l'on cherche,
par exemple \giacinputmath{assume(m>1)} puis \giacinputmath{solve(m^2-4=0,m)}.
\item \verb|solve| permet aussi de r\'esoudre des syst\`emes
polynomiaux simples, on donne en 1er argument la liste des
\'equations, en 2\`eme argument la liste des variables.
Par exemple intersection d'un cercle et d'une droite
\giacinputmath{solve([x^2+y^2+2y=3,x+y=1],[x,y])}
\item \verb|linsolve| permet de r\'esoudre des syst\`emes
lin\'eaires. On lui passe la liste des \'equations et la liste
des variables (par convention une expression \'equivaut
\`a l'\'equation expression=0). Par exemple
\giacinputmath{linsolve([x+2y=3,x-y=7],[x,y])}
\verb|linsolve| renvoie la solution g\'en\'erale du syst\`eme 
(y compris si la solution n'est pas unique).
\item \verb|desolve| permet de r\'esoudre de mani\`ere exacte
certaines \'equations diff\'erentielles, par exemple 
pour r\'esoudre $y'=2y$, on tape \verb|desolve(y'=2y)|.\\
Un exemple o\`u on indique une condition initiale, 
la variable ind\'ependante
et la fonction inconnue~:\\
\verb|desolve([y'=2y,y(0)=1],x,y)|
Utiliser \verb|odesolve|
pour une r\'esolution approch\'ee et \verb|plotode| pour une
repr\'esentation graphique de solution calcul\'ee de mani\`ere
approch\'ee. 
\item \verb|rsolve| permet de r\'esoudre de mani\`ere exacte
certaines relations de r\'ecurrences $u_{n+1}=f(u_n,...)$,
par exemple les suites arithm\'etico-g\'eom\'etriques, par
exemple $u_{n+1}=2u_n+3, u_0=1$
\giacinputmath{rsolve(u(n+1)=2*u(n)+3,u(n),u(0)=1)}
\end{itemize}

\subsection{Arithm\'etique}
Lorsque cela est n\'ecessaire,
on distingue l'arithm\'etique des entiers de celle des polyn\^omes
par l'existence du pr\'efixe \verb|i| (comme \verb|integer|) dans
un nom de commande, par exemple \verb|ifactor| factorise un entier
(pas trop grand) alors que \verb|factor| factorise un polyn\^ome
(et \verb|cfactor| factorise un polyn\^ome sur les complexes).
Certaines commandes fonctionnent \`a la fois pour les entiers et
les polyn\^omes, par exemple \verb|gcd| et \verb|lcm|.

\subsubsection{Entiers}
Depuis le catalogue, s\'electionner le sous-menu \verb|Arithmetic, Crypto|
\begin{itemize}
\item \verb|iquo(a,b)|, 
\verb|irem(a,b)| quotient et reste de la division euclidienne
de deux entiers.\\
\giacinputmath{iquo(23,13),irem(23,13)}
\item \verb|isprime(n)| teste si $n$ est un nombre premier.
Le test est probabiliste pour de grandes valeurs de $n$.
\giacinputmath{isprime(2^64+1)}
\item \verb|ifactor(n)| factorise un entier pas trop grand
(les algorithmes utilis\'es se limitent \`a la division et Pollard-$\rho$,
il n'y avait pas de place pour le crible quadratique).
Par exemple
\giacinputmath{ifactor(2^64+1)}\\
Raccourci clavier touches $\rightarrow$ puis * (\verb|=>*|)
\item \verb|gcd(a,b)|, \verb|lcm(a,b)| PGCD et PPCM de deux entiers
ou de deux poln\^omes\\
\giacinputmath{gcd(25,15),lcm(25,15)}\\
\giacinputmath{gcd(x^3-1,x^2-1),lcm(x^3-1,x^2-1)}
\item \verb|iegcd(a,b)| renvoie 3 entiers $u,v,d$ tels que
$au+bv=d$ o\`u $d$ est le PGCD de $a$ et $b$, avec $|u|<|b|$ et
$|v|<|a|$.\\
\giacinputmath{u,v,d:=iegcd(23,13); 23u+13v}
\item \verb|ichinrem([a,m],[b,n])| lorsque cela est possible,
renvoie $c$ tel que $c=a \pmod m$ et $c=b \pmod n$ (si $m$ et $n$
sont premiers entre eux, $c$ existe).
\giacinputmath{c,n:=ichinrem([1,23],[2,13]); irem(c,23); irem(c,13)}
\item \verb|powmod(a,n,m)| calcule $a^n \pmod m$ par
l'algorithme de la puissance rapide modulaire.
\giacinputmath{powmod(7,22,23)}
\end{itemize}
Les commandes \verb|asc| et \verb|char| permettent de convertir
une chaine de caract\`eres en liste d'entiers (entre 0 et 255) et
r\'eciproquement, ce qui permet de faire facilement de la cryptographie
avec des messages sous forme de chaines de caract\`eres.

\subsubsection{Polyn\^omes}
Depuis le catalogue, s\'electionner le sous-menu \verb|Polynomes|.
La variable est par d\'efaut $x$, sinon il faut la sp\'ecifier en
g\'en\'eral en dernier argument, par exemple \verb|degree(x^2*y)|
ou \verb|degree(x^2*y,x)| renvoient 2, alors que \verb|degree(x^2*y,y)|
renvoie 1
\begin{itemize}
\item \verb|coeff(P,n)| coefficient de $x^n$ dans $P$, \verb|lcoeff(P)|
coefficient dominant de $P$, par exemple
\giacinputmath{P:=x^3+3x; coeff(P,1); lcoeff(P)}
\item \verb|degre(P)| degr\'e du polyn\^ome $P$.
\giacinput{degree(x^3)}
\item \verb|quo(P,Q)|, \verb|rem(P,Q)| quotient et reste de la division
euclidienne de \verb|P| par \verb|Q|
\giacinputmath{P:=x^3+7x-5; Q:=x^2+x; quo(P,Q); rem(P,Q)}
\item \verb|proot(P)|~: racines approch\'ees de $P$ (r\'eelles et complexes)\\
\giacinput{proot(x^5+x+1)}
Repr\'esentation graphique~:\\
\giacinput{point(proot(x^5+x+1))}
\item \verb|interp(X,Y)|~: pour deux listes
de m\^eme taille, polyn\^ome d'interpolation
passant par les points $(X_i,Y_i)$. \\
\giacinputmath{X,Y:=[0,1,2,3],[1,-3,-2,0]; P:=interp(X,Y)=>+}
Repr\'esentation graphique\\
\giacinput{scatterplot(X,Y); plot(P,x,-1,4)}
\item \verb|resultant(P,Q)|~: r\'esultant des polyn\^omes
$P$ et $Q$
\giacinput{P:=x^3+7x-5; Q:=x^2+x; resultant(P,Q)}
\item \verb|hermite(x,n)|~: n-i\`eme polyn\^ome de Hermite,
orthogonal pour la densit\'e $e^{-x^2}dx$ sur $\R$
\item \verb|laguerre(x,n,a)|~: n-i\`eme polyn\^ome de Laguerre,
\item \verb|legendre(x,n)|~: n-i\`eme polyn\^ome de Legendre,
orthogonal pour la densit\'e $dx$ sur $[-1,1]$
\item \verb|tchebyshev1(n)| et \verb|tchebyshev2(n)|
polyn\^omes de Tchebyshev de 1\`ere et 2\`eme esp\`ece
d\'efinis par~:
$$T_n(\cos(x))=\cos(nx), \quad U_n(\cos(x))\sin(x)=\sin((n+1)x)$$
%\item \verb|bernoulli(n,x)| polyn\^ome de Bernoulli
\end{itemize}

\subsection{Alg\`ebre lin\'eaire, vecteurs, matrices}
Xcas ne fait pas de diff\'erence entre vecteur et liste. Par
exemple pour faire le produit scalaire de deux vecteurs, on peut
saisir~:
\begin{giaconload}v:=[1,2]; w:=[3,4]\end{giaconload}
\giacinput{dot(v,w)}

Pour saisir une matrice \'el\'ement par \'el\'ement, taper
sur shift-MATR et s\'electionner \verb|matrix| 
ou F6, puis 10 (editer matrice). Vous pouvez ensuite
cr\'eer une nouvelle matrice ou \'editer
une matrice existante parmi la liste de variables propos\'ees. 
Notez que la touche \verb|,| permet d'ins\'erer une ligne ou une colonne,
alors que la touche d'effacement \verb|DEL| efface la ligne ou la colonne de la
s\'election (tapez \verb|UNDO| si vous avez effac\'e par erreur
une ligne ou colonne).
Pour de petites matrices, vous pouvez aussi
entrer en ligne de commandes
une liste de listes de m\^eme taille. Par exemple
pour d\'efinir la matrice
$$ A=\left(\begin{array}{cc} 1 & 2 \\ 3 & 4 \end{array}\right)$$
on peut taper \begin{giaconload}A:=[[1,2],[3,4]]\end{giaconload} 
ou \giacinputmath{[[1,2],[3,4]]=>A}

Il est fortement conseill\'e de stocker les matrices dans des variables
pour \'eviter de les saisir plusieurs fois.

Pour entrer une matrice dont les coefficients sont
donn\'es par une formule d\'efinissant l'\'el\'ement $a_{ij}$
en fonction de la ligne $i$ et colonne $j$,
on peut utiliser la commande \verb|matrix(|.
Par exemple 
\giacinputmath{matrix(2,2,(j,k)->1/(j+k+1))} 
renvoie
la matrice dont le coefficient ligne $j$ et colonne $k$ vaut
$\frac{1}{j+k+1}$ (attention les indices commencent \`a 0). 
Pour obtenir la commande \verb|matrix(|, taper F4 (cmds) puis
17 Matrices ou utiliser le raccourci shift-MATR, EXE, \verb|AC/ON|.

La matrice identit\'e de taille $n$ 
est renvoy\'ee par la commande \verb|idn(n)| (F4 17),
alors que \verb|ranm(n,m,loi,[parametres])| (menu shift-MATR ou F4 17)
renvoie une matrice \`a coefficients al\'eatoires de taille $n,m$.
Par exemple 
\giacinputmath{U:=ranm(4,4,uniformd,0,1)}
\giacinputmath{N:=ranm(4,4,normald,0,1)}

Pour ex\'ecuter une commande sur des matrices, s'il s'agit
d'arithm\'etique de base (\verb|+,-,*| inverse), on utilise
les op\'erations au clavier. Pour les autres commandes.
depuis le catalogue F4, s\'electionner le sous-menu \verb|Matrices|
\begin{itemize}
\item \giacinputmath{eigenvals(A)}
\giacinputmath{eigenvects(A)} 
renvoient les valeurs
propres et vecteurs propres d'une matrice carr\'ee $A$. 
\item \giacinputmath{P,D:=jordan(A)} 
calcule la forme normale de Jordan d'une matrice $A$
(\`a coefficients exacts) et renvoie les matrices $P$ et $D$ telles que
$P^{-1}AP=D$, avec $D$ triangulaire sup\'erieure (diagonale si $A$ est
diagonalisable)
\item \giacinputmath{Ak:=matpow(A,k)} calcule la puissance $k$-i\`eme
d'une matrice $A$ avec $k$ une variable formelle.
\item \verb|rref| effectue la r\'eduction sous forme \'echelonn\'ee
d'une matrice $A$ (pivot de Gauss)
\item \verb|lu| calcule la d\'ecomposition $LU$ d'une matrice $A$
et renvoie une permutation de matrice $P$ et deux matrices $L$
triangulaire inf\'erieure et $U$ triangulaire sup\'erieure telles que
$PA=LU$. Le r\'esultat de la commande 
\giacinput{P,L,U:=lu(A)} peut
\^etre pass\'e en argument \`a la commande \giacinput{linsolve(P,L,U,v)}
pour r\'esoudre un syst\`eme $Ax=b$ de matrice $A$ en r\'esolvant
deux syst\`emes triangulaires (calcul en $O(n^2)$ au lieu
de $O(n^3)$).
\item \verb|qr| calcule la d\'ecomposition $QR$ d'une matrice $A$
et renvoie deux matrices $Q$ orthogonale et $R$ triangulaire sup\'erieure
telles que $A=QR$.
\item \verb|svd(A)| calcule la factorisation en valeurs
singuli\`eres d'une matrice $A$, et renvoie 
$U$ orthogonale, $S$ vecteur des valeurs singulières, $Q$
orthogonale tels que \verb|A=U*diag(S)*tran(Q)|. Le rapport
de la plus grande valeur singuli\`ere de $S$ sur la plus petite
donne le nombre de condition de $A$ relativement \`a la norme euclidienne,
plus ce nombre est grand, plus on perd en pr\'ecision en r\'esolvant
un syst\`eme $Ax=b$ lorsque $b$ n'est pas connu exactement.
\end{itemize}


\section{Probabilit\'es et statistiques}

\subsection{Tirages al\'eatoires}
Depuis le catalogue F4, s\'electionner le sous-menu \verb|Probabilites|
puis s\'electionnez \giacinput{rand()} (r\'eel selon
la loi uniforme dans $[0,1]$) ou 
\giacinput{n:=6:; randint(n)} 
(entier
entre 1 et $n$). De nombreuses autres fonctions al\'eatoires
existent, avec comme pr\'efixe \verb|rand|, suivi par le nom
de la loi, par exemple \verb|randbinomial(n,p)| renvoie
un entier al\'eatoire selon la loi binomiale de param\`etres $n,p$.
Pour cr\'eer un vecteur ou une matrice al\'eatoire, utiliser la commande
\verb|ranv| ou \verb|ranm| (menu F4 \verb|Matrice| ou shift-MATR), 
par exemple pour un vecteur de 10 composantes
selon la loi normale centr\'ee r\'eduite 
\giacinputmath{ranv(10,normald,0,1)}

\subsection{Lois de probabilit\'es}
Depuis F4, s\'electionner le sous-menu \verb|Probabilites|.
Les lois propos\'ees dans le catalogue sont la loi binomiale, la loi normale,
la loi exponentielle et la loi uniforme. 
D'autres lois sont disponibles mais il n'y a pas assez de place pour
les documenter~:
\verb|chisquared|, \verb|geometric|, \verb|multinomial|
\verb|studentd|, \verb|fisherd|, \verb|poisson|.

Pour obtenir la distribution cumul\'ee d'une loi, on saisit le nom de la loi
et le suffixe \verb|_cdf| (s\'electionner \verb|cdf| dans le catalogue
sous-menu Probabilit\'es et taper F1). 
Pour obtenir la distribution cumul\'ee inverse,
on saisit le nom de la loi et le suffixe \verb|_icdf|  
(s\'electionner \verb|cdf| dans le catalogue
sous-menu Probabilit\'es et taper F2).

Exemple~: calcul de l'intervalle centr\'e $I$
pour la loi normale de moyenne 5000 et d'\'ecart-type 200
tel que la probabilit\'e d'\^etre en-dehors de $I$ soit de 5\%~:
\giacinput{M:=5000; S:=200; normald_icdf(M,S,0.025);normald_icdf(M,S,0.975)}

\subsection{Statistiques descriptives 1-d}
Ces fonctions agissent sur des listes
par exemple \begin{giaconload}l:=[9,11,6,13,17,10]\end{giaconload}
Depuis le catalogue, s\'electionner le sous-menu \verb|Statistiques|.
\begin{itemize}
\item \giacinput{mean(l)}~: moyenne arithm\'etique d'une liste
\item \giacinput{stddev(l)}~: \'ecart-type d'une liste\\
Utiliser
\giacinput{stddevp(l)} pour avoir un estimateur non biais\'e
de l'\'ecart-type d'une population dont \verb|l| est un \'echantillon
\item \giacinput{median(l)}, \giacinput{quartile1(l)}, 
\giacinput{quartile3(l)}
renvoient respectivement la m\'ediane, le 1er et 3\`eme quartiles
d'une liste
\end{itemize}
Pour les statistiques 1-d de listes avec effectifs, on remplace
\verb|l| par deux listes de m\^eme longueur, la 1\`ere liste
est la liste des valeurs de la s\'erie statistique, la 2\`eme
liste est la liste des effectifs.
Voir aussi les commandes du menu \verb|Graphique| 
\verb|histogram| et \verb|barplot|.

\subsection{Statistiques descriptives 2-d}
Depuis le catalogue, s\'electionner le sous-menu \verb|Statistiques|.
\begin{itemize}
\item \verb|correlation(X,Y)| calcule la corr\'elation entre 
2 listes de m\^eme taille.
\item \verb|covariance(X,Y)| calcule la covariance entre 
2 listes de m\^eme taille.
\item les commandes de suffixe \verb|_regression(X,Y)| calculent
des ajustements par r\'egression au sens des moindres carr\'es,
par exemple \verb|linear_regression(X,Y)| renvoie
les coefficients $m,p$ de la droite de r\'egression lin\'eaire
$y=mx+p$.\\
\item \verb|linear_regression_plot(X,Y)| (respectivement
 toutes les commandes
de suffixe \verb|_regression_plot|) trace une droite (respectivement
une courbe) d'ajustement des donn\'ees contenues dans les listes
\verb|X,Y| de m\^eme taille.
Ces commandes affichent
de plus le coefficient $R^2$ qui permet de quantifier la qualit\'e
de l'ajustement (plus $R^2$ est proche de 1, meilleur est
l'ajustement).
\end{itemize}

\section{Graphes}
Pour obtenir une repr\'esentation graphique, on saisit une commande
de trac\'e (ou plusieurs commandes s\'epar\'ees par \verb|;|).
Depuis le catalogue F4, s\'electionner le sous-menu \verb|Graphes|
(7).
\begin{itemize}
\item \verb|plot(f(x),x=a..b)| graphe de $f(x)$ pour $x\in [a,b]$.
On peut sp\'ecifier un pas de discr\'etisation avec \verb|xstep=|,
par exemple \giacinput{plot(x^2,x=-4..4,xstep=1)}
\item \verb|plotseq(f(x),x=[u0,a,b])| graphe ``en toile d'araign\'ee''
de la suite r\'ecurrente $u_{n+1}=f(u_n)$ de premier terme $u_0$ donn\'e.
Par exemple si $u_{n+1}=\sqrt{2+u_n}, u_0=6$ avec une repr\'esentation
sur $[0,7]$
\giacinput{plotseq(sqrt(2+x),x=[6,0,7])}
\item \verb|plotparam([x(t),y(t)],t=tm..tM)| courbe en param\'etriques
$(x(t),y(t))$ pour $t\in [t_m,t_M]$.
On peut sp\'ecifier un pas de discr\'etisation avec \verb|tstep=|
par exemple
\giacinput{plotparam([sin(2t),cos(3t)],t,0,2*pi)}
%On peut sp\'ecifier la fen\^etre graphique avec
\item \verb|plotpolar(r(theta),theta=a..b)| courbe
en polaires $r(\theta)$ pour $\theta \in [a,b]$,
par exemple
\giacinput{plotpolar(sin(3*theta),theta,0,2*pi)}
\item \verb|plotlist(l)| pour une liste \verb|l|,
trace la ligne polygonale reliant
les points de coordonn\'ees $(i,l_i)$ (indice $i$ commen\c{c}ant \`a 0).\\
\verb|plotlist([X1,Y1],[X2,Y2],...)| 
trace la ligne polygonale reliant
les points de coordonn\'ees $(X_i,Y_i)$ 
\item \verb|scatterplot(X,Y)|, \verb|polygonscatterplot(X,Y)|
pour 2 listes \verb|X,Y| de m\^eme taille, trace un nuage de points
ou une ligne polygonale reliant
les points de coordonn\'ees $(X_i,Y_i)$ 
\item \verb|histogram(l,class_min,class_size)| trace
l'histogramme des donn\'ees de la liste \verb|l| avec
comme largeur de classe \verb|class_size| en commen\c{c}ant
\`a \verb|class_min|. Par exemple, on peut tester la qualit\'e
du g\'en\'erateur al\'eatoire avec
\giacinput{l:=ranv(500,normald,0,1); histogram(l,-4,0.25); plot(normald(x),x,-4,4)}
\item \verb|plotcontour(f(x,y),[x=xmin..xmax,y=ymin..ymax],[l0,l1,...])|
trace les courbes de niveaux $f(x,y)=l_0, f(x,y)=l_1, ...$.
\item (Graph 90)
\verb|plotfield(f(t,y),[t=tmin..tmax,y=ymin..ymax])|
trace le champ des tangentes \`a l'\'equation diff\'erentielle
$y'=f(t,y)$. On peut ajouter en dernier param\`etre
optionnel \verb|,plotode=[t0,y0]|
pour tracer simultan\'ement la solution passant par la condition
initiale $y(t_0)=y_0$. Exemple $y'=\sin(ty)$ sur l'intervalle $[-3,3]$
en temps et $[-2,2]$ en $y$\\
\giacinput{plotfield(sin(t*y),[t=-3..3,y=-3..3],plotode=[0,1])}
On peut aussi utiliser la commande \verb|plotode| en-dehors d'une commande
\verb|plotfield|.\\
Sur la graph 35eii, 
la m\'emoire est insuffisante pour utiliser \verb|plotfield|
avec l'option \verb|plotode|.
\item On peut tracer simultan\'ement plusieurs graphiques, en s\'eparant
les commandes de trac\'e par \verb|;|
\end{itemize}
Lorsqu'un graphique comporte des trac\'es de courbes (graphes de
fonction, courbes en param\'etriques ou en polaires), le mode
``trace'' est activ\'e par d\'efaut sur $\chi$CAS en 2 addins pour Graph 90
et sur $\chi$CAS pour Graph 35eii. Dans ce mode, l'appui sur la
touche curseur vers la droite ou la gauche permet de d\'eplacer
un pointeur sur la courbe active, en affichant les coordonn\'ees
du pointeur (et du param\`etre pour une courbe en param\'etriques)
et d'un vecteur tangent. S'il y a plusieurs arcs de courbe, les
touches de curseur haut et bas permettent de changer l'arc d'\'etude.
L'appui sur F2 permet d'afficher les informations sur la courbe
courante, si on confirme par EXE on acc\`ede \`a un tableau de valeurs.

Le menu F1 \'etude de graphes (raccourci touche x,$\theta$,t)
permet d'ajouter un vecteur normal pointant vers le centre de courbure
(F4),
ou/et le cercle osculateur (F5) et le rayon de courbure. On peut
aussi \`a partir de ce menu d\'eplacer le pointeur vers une position d\'etermin\'ee, ou
vers un point remarquable: racine, tangente horizontale ou verticale,
point d'inflexion, intersection avec un autre arc de courbe 
(fonctionnalit\'e limit\'ee \`a l'intersection de graphes de fonctions 
sur la 35eii par manque de place). On peut
enfin calculer une longueur d'arc ou l'aire sous la courbe entre
le pointeur et la marque. 

Lorsque vous faites une \'etude de courbe, les variables
\verb|X0,X1,X2,Y0,Y1,Y2|
sont automatiquement affect\'ees avec l'expression de la position, vitesse
et acc\'el\'eration sur la courbe \'etudi\'ee. Lors de la recherche
d'une racine, tangente horizontale, inflexion, longueur d'arc, aire
sous la courbe, des variables contenant la derni\`ere recherche
sont cr\'ees.

La touche OPTN vous permet de sp\'ecifier certaines options graphiques~:
\begin{itemize}
\item Pour modifier les couleurs des graphes, utiliser la touche OPTN
et s\'electionnez une couleur puis tapez F2 (Exemple 1) pour ins\'erer
\verb|display=|{\em couleur}, par exemple OPTN r e d F2 pour
\giacinput{plot(sin(x),display=red)}
\item Pour changer l'\'epaisseur des segments (y compris les lignes
polygonales utilis\'ees pour tracer une courbe), utiliser
\verb|display=line_width_2| \`a \verb|display=line_width_8|.
OPTN l i puis curseur vers le bas si n\'ecessaire puis F2
Pour changer \`a la fois la couleur et l'\'epaisseur, additionnez les
attributs, par exemple~:
\verb|display=red+line_width_2|
\item Les cercles ainsi que les polygones 
peuvent \^etre remplis avec l'attribut
\verb|display=filled| (qui peut s'additionner \`a d'autres attributs).
\item Pour remplacer la fen\^etre graphique calcul\'ee automatiquement
par des valeurs pr\'ed\'efinies, utiliser la touche OPTN,
s\'electionner \verb|gl_x| ou/et \verb|gl_y| et indiquer l'intervalle
en $x$ ou en $y$ souhait\'e, la commande doit pr\'ec\'der une
commande de trac\'e. Par exemple 
\giacinput{gl_x=-2..2;gl_y=-1..4;plot(exp(x))}
\item
Pour enlever les axes, s\'electionner \verb|axes| puis taper F2 
(\verb|axes=0|). La commande doit pr\'ec\'eder une commande de
trac\'e.
\end{itemize}

\section{L'\'editeur d'expressions}
Lorsqu'un calcul renvoie une expression, elle est affich\'ee en plein 
\'ecran dans
l'\'editeur d'expression 2d. Depuis l'historique des calculs, si le niveau
s\'electionn\'e est une expression, l'appui sur F3 (voir) affiche
l'expression dans l'\'editeur 2d. En ligne de commande, l'appui
sur F3 (voir) ouvre aussi l'\'editeur 2d, soit avec 0 si la ligne
de commande \'etait vide, ou avec le contenu de la ligne de commande si
celle-ci est syntaxiquement correcte.

Lorsque l'\'editeur 2d est ouvert, l'expression est affich\'ee en
plein \'ecran et une
partie de l'expression est s\'electionn\'ee. On peut alors agir sur
la s\'election en ex\'ecutant des commandes saisies via les menus
ou le clavier, on peut aussi \'editer la s\'election (en mode de saisie
1d). Ceci permet de retravailler des sous-expressions
ou d'\'editer une expression en \'ecriture naturelle.

Exemple 1~: nous allons saisir 
$$ \lim_{x \rightarrow 0} \frac{\sin(x)}{x}$$
Depuis une ligne de commande vide, taper F3 (voir), vous devez voir 0 
s\'electionn\'e. Tapez sur la touche x et EXE, maintenant c'est x qui
est en surbrillance. Tapez sur la touche SIN, c'est $\sin(x)$ qui
est en surbrillance. Tapez sur la touche de division (au-dessus de -), 
vous devez voir $\frac{\sin(x)}{0}$ avec 0 en surbrillance, tapez
x puis EXE, vous devez voir $\frac{\sin(x)}{x}$ avec
x au d\'enominateur en surbrillance. Tapez sur le curseur
fl\`eche vers le haut pour mettre $\frac{\sin(x)}{x}$ en surbrillance,
puis F2 4 (pour limit). L'expression est correcte, vous pouvez taper EXE
pour la recopier en ligne de commande and EXE again to eval it. 
Si on avait voulu
une limite en $+\infty$, il aurait fallu d\'eplacer la s\'election
avec curseur vers la droite, puis faire F1 8 (oo) EXE.

Exemple 2~: nous allons saisir
$$ \int_0^{+\infty} \frac{1}{x^4+1} \ dx $$
Depuis une ligne de commande vide, taper F3 (voir), puis F2 3 (integrate),
vous devez voir
$$ \int_0^1 0 \ dx$$
avec $x$ s\'electionn\'e. Il faut donc changer le 1 de la borne sup\'erieure
et le 0 \`a int\'egrer. Pour modifier le 0, curseur vers la gauche pour
le s\'electionner puis \verb|1/(x^4+1)| EXE, puis curseur vers
la gauche F1 8 EXE. Taper sur EXE pour recopier vers la ligne
de commande puis EXE pour effectuer le calcul, le r\'esultat s'affiche
dans l'\'editeur 2d, EXE quitte l'\'editeur avec dans l'historique
l'int\'egrale et sa valeur (en syntaxe alg\'ebrique 1d).

Exemple 3~: nous allons calculer et simplifier 
$$ \int \frac{1}{x^4+1} \ dx $$
Depuis une ligne de commande vide, taper F3 (voir), puis F2 3 (integrate),
vous devez voir
$$ \int_0^1 0 \ dx$$
D\'eplacez le curseur sur le 0 de la borne inf\'erieure de l'int\'egrale
et tapez sur la touche DEL, vous devez voir
$$ \int 0 \ dx$$
avec le tout s\'electionn\'e. Utilisez le curseur vers le bas
pour s\'electionner 0 et tapez \verb|1/(x^4+1)| EXE puis EXE pour
recopier en ligne de commande puis EXE pour ex\'ecuter le calcul,
le r\'esultat s'affiche maintenant dans l'\'editeur 2d.\\
On peut alors s\'electionner avec les touches du curseur par exemple
l'argument d'un des arctangentes et ex\'ecuter F1 EXE (simplify) 
pour effectuer une simplification partielle du r\'esultat, puis
recommencer avec l'autre arctangente.\\
On peut simplifier encore plus, en rassemblant les logarithmes. Pour
cela il faut d'abord \'echanger deux des arguments de la somme. 
S\'electionnez un des logarithmes avec des d\'eplacements du curseur,
puis tapez
\begin{itemize}
\item Graph 90: shift-curseur droit ou gauche
\item Graph 35: F5 curseur droit ou gauche, ensuite ALPHA
\end{itemize}
cela \'echange l'argument s\'electionn\'e avec son fr\`ere de droite ou
de gauche. Tapez ensuite 
 ALPHA curseur vers la droite ou vers la gauche, ceci augmente la
s\'election en ajoutant le fr\`ere de droite ou de gauche.
Une fois les deux logarithmes s\'electionn\'es, menu F1 2 EXE (factor), puis
descendez la s\'election sur la somme ou diff\'erence de logarithmes,
allez dans le menu F4 (cmds) puis EXE (Tout), tapez les lettres l, n, c
ce qui d\'eplace \`a la premi\`ere commande commen\c{c}ant par \verb|lnc|,
s\'electionnez \verb|lncollect|, validez et tapez enfin sur F6 (eval).

\section{Sessions de calculs}
\subsection{Edition de l'historique.}
En utilisant la touche curseur vers le haut/bas, on se d\'eplace dans
l'historique des calculs, le niveau courant est en surbrillance. 

Pour modifier l'ordre des niveaux dans l'historique des calculs,
tapez ALPHA-curseur vers le haut ou vers le bas.
Pour effacer un niveau, appuyez sur la touche DEL (le niveau
est recopi\'e dans le presse-papiers).

Pour
modifier un niveau existant, on tape sur F3 ou sur ALPHA-F3. Dans le
premier cas, c'est l'\'editeur 2d qui est appel\'e si le niveau
est une expression, dans le
deuxi\`eme cas, c'est l'\'editeur texte qui est appel\'e.
Taper EXIT pour annuler les modifications ou EXE pour valider.
Si les modifications sont valid\'ees, les lignes de commande situ\'ees
en-dessous de la ligne modifi\'ee seront automatiquement calcul\'ees,
tenant compte des modifications, par exemple si vous modifiez un niveau
comme \verb|A:=1|, les lignes situ\'ees en-dessous d\'ependant de \verb|A|
seront actualis\'ees.

Ce processus peut \^etre automatis\'e en utilisant un curseur,
que l'on peut cr\'eer avec un assistant, depuis le menu F6,
Parameter. Une fois cr\'e\'e, vous pouvez modifier un curseur
en tapant sur les touches + ou - lorsque le niveau contenant
la commande \verb|assume| ou \verb|parameter| est s\'electionn\'e
(tapez * ou / pour une modification plus rapide).

\subsection{Variables}
En appuyant sur la touche \verb|VARS| vous affichez la liste
des variables qui ont une valeur, ainsi que des commandes de gestion
de variables. D\'eplacez le curseur vers une variable puis EXE pour
la recopier en ligne de commande, DEL copie en ligne de commande
la commande d'effacement de la variable (confirmez ensuite avec EXE).
La commande \verb|restart| permet d'effacer toutes les variables.
La commande \verb|assume| permet de faire une hypoth\`ese sur
une variable, par exemple \verb|assume(x>5)| (\verb|>| se trouve
dans le menu shift-PRGM).

\subsection{Sauvegarde et \'echange de sessions avec Xcas.}
Depuis l'historique des calculs, avec le menu F6, vous pouvez sauvegarder
une session de travail sur la m\'emoire de stockage de la
calculatrice. Les fichiers ont comme extension \verb|xw|.
Vous pouvez alors sauvegarder ces fichiers sur votre ordinateur en
connectant la calculatrice comme clef USB (choix F1), depuis
l'explorateur de fichiers de votre ordinateur.

Ces fichiers sont compatibles avec Xcas et Xcas pour Firefox. Vous
pouvez ouvrir une session depuis le menu Fich, Ouvrir de Xcas
ou depuis le bouton Charger/Parcourir de la page de 
\ahref{https://www-fourier.univ-grenoble-alpes.fr/\home{parisse}/xcasfr.html}{Xcas pour Firefox}.
Vous pouvez sauvegarder des sessions Xcas et Xcas pour Firefox au format
\verb|xw| dans le menu Fich, Exporter comme Khicas de Xcas, 
ou le bouton Export de Xcas pour Firefox.



\section{Programmation}
Vous pouvez programmer en utilisant
les structures de commande en fran\c{c}ais de Xcas
ou en utilisant la compatibilit\'e de
syntaxe Python. Les programmes tr\`es courts (en une ligne)
peuvent \^etre saisis directement en ligne de commande. Les
programmes plus longs ou que l'on souhaite sauvegarder
seront saisis dans l'\'editeur de programmes sur
la calculatrice, ou bien transf\'er\'es depuis un PC ou
une autre calculatrice.

\subsection{Utilisation en exemples}
{\bf Un premier exemple en ligne de commande~}:\\
une fonction d\'efinie par une expression alg\'ebrique. On
saisit \verb|nom_fonction(parametres):=expression|
Par exemple, pour d\'efinir le p\'erim\`etre d'un cercle de
rayon $r$, on peut taper
\giacinput{peri(r):=2*pi*r}
puis on peut calculer \giacinput{peri(1)}.
Pour saisir \verb|:=| on peut utiliser le raccourci
shift-$\rightarrow$ ou shift-INS selon la version de $\chi$CAS.

{\bf Autre exemple, pour calculer l'intervalle de confiance de seconde}
connaissant une fr\'equence $p$ et un effectif $n$, on tape\\
\giacinput{F(P,N):=[P-1/sqrt(N),P+1/sqrt(N)]}\\
puis on teste \giacinput{F(0.4,30)}

{\bf Autre exemple~: avec la tortue de Xcas}\\
La tortue de Xcas est un petit robot qui se d\'eplace selon des 
ordres qui lui sont donn\'es en laissant une trace de son passage.
Les commandes de la tortue sont accessibles depuis le dernier item
du menu F6, ou par le raccourci shift-QUIT.
Saisir la commande {\tt avance} dans ce menu puis valider, vous
devez voir la tortue (symbolis\'ee par un triangle) avancer de 10 pixels.
Taper EXIT pour revenir en ligne de commande.
Saisir la commande \verb|tourne_gauche| et valider, la tortue a tourn\'e
de 90 degr\'es. R\'ep\'eter
3 fois ces deux commandes pour afficher un carr\'e.\\
Pour effacer le dessin et ramener la tortue \`a l'origine,
saisir la commande {\tt efface}.
Pour faire des dessins tortue, il est conseill\'e d'utiliser l'\'editeur
de programmes (cf. ci-dessous).

{\bf Autre exemple~: une boucle ``oneliner'' en syntaxe Xcas.}\\
Tapez shift-PRGM, puis s\'electionnez l'exemple
de \verb|pour (boucle Xcas)|\\
\giacinput{pour j de 1 jusque 10 faire print(j,j^2); fpour;}
tapez sur EXE, vous devez voir les carr\'es des entiers de 1 \`a 10.

{\bf Exercice~:} faire faire un carr\'e \`a la tortue en utilisant
une boucle.

{\bf Utilisation de l'\'editeur}\\
Modifions cet exemple pour faire afficher les carr\'es de 1 \`a $n$
en utilisant la syntaxe compatible Python et
l'\'editeur de programmes. V\'erifiez que la syntaxe Python
est coch\'ee (F6 ou shift-SETUP), sinon cochez-la.
Tapez F6 (Fich,Cfg) puis s\'electionner
Editeur script
pour ouvrir l'\'editeur. S'il n'y a pas encore de script,
on vous propose de choisir entre Tortue ou Programme, s\'electionnez
Programme, vous devriez obtenir
\verb|def session(x):|. Remplacez \verb|session| par \verb|f| et
\verb|x| par \verb|n|, puis 
d\'eplacez le curseur en fin de ligne et passez \`a la
ligne (shift-EXE).
Tapez Shift-PRGM puis \verb|1 for|, placez le curseur entre \verb|for|
et \verb|in| et tapez F5 j, placez le curseur entre les parenth\`eses
de \verb|range| tapez \verb|1,n+1| puis
d\'eplacez le curseur \`a la ligne suivante
F4 (cmds), EXE (\verb|1 Tout|), P, R, d\'eplacez
la surbrillance avec le curseur vers le bas jusque \verb|print| 
et validez (EXE), puis tapez \verb|j,j^2)| (F5 j ALPHA , F5 j ALPHA $x^2$). 
Vous devriez avoir le  programme suivant~:
\begin{giacprog}
def f(n):
  for j in range(1,n+1):
    print(j,j**2)
  return 
\end{giacprog}
N.B.: pour la puissance, on peut utiliser \verb|^| ou \verb|**| dans 
KhiCAS (sauf si on a ex\'ecut\'e la commande \verb|python_compat(2)|),
notez qu'il faut utiliser \verb|**| en Python.

Maintenant, tapez F6 (Fich,Cfg) et s\'electionnez \verb|1. Tester syntaxe|
ou tapez EXE.
Si tout va bien, vous devez voir \verb|Syntaxe correcte| 
dans la ligne d'\'etat.
Sinon, le num\'ero de ligne de la premi\`ere erreur est indiqu\'e ainsi que
le mot qui a provoqu\'e l'erreur. Le curseur est positionn\'e sur la ligne
o\`u l'erreur a \'et\'e d\'etect\'ee (il peut arriver que l'erreur soit 
situ\'ee avant mais d\'etect\'ee un peu plus loin seulement). Si vous
utilisez la syntaxe en Python, notez que les structures de programmation
sont traduites en langage Xcas, les erreurs affich\'ees le sont par
rapport \`a cette traduction (donc des mots-clefs de fin de structure
comme \verb|end| rajout\'es par le traducteur
peuvent apparaitre dans le message d'erreur).

Si le programme est syntaxiquement correct, vous pouvez le sauvegarder
depuis le menu F6 (File). Pour l'ex\'ecuter, revenez \`a la
ligne de commande en tapant la touche EXIT. En tapant la touche
VARS, \verb|f| doit apparaitre, s\'electionnez \verb|f|
puis compl\'etez pour obtenir par
exemple la commande \verb|f(10)|, puis EXE, vous devriez voir s'afficher
les carr\'es de 1 \`a 10.


{\bf 3i\`eme exemple~: Calcul de l'intervalle de confiance de terminale S}\\
On peut le saisir en ligne de commande
\giacinput{F(P,N):=[P-1.96*sqrt(P*(1-P)/N),P+1.96*sqrt(P*(1-P)/N)]}
On peut \'eviter les calculs redondants en utilisant une variable
locale (utiliser shift-PRGM pour saisir def et return)\\
\begin{giacprog}
def F(P,N):
  D=1.96*sqrt(P*(1-P)/N)
  return [P-D,P+D]
\end{giacprog}

{\bf Exercice~} Cr\'eez une nouvelle session  contenant
un script pour afficher un carr\'e avec la tortue.
Cr\'eez ensuite une fonction \verb|f| d'argument $n$
pour afficher un carr\'e de $n$ pixels.

{\bf Solution~}
Faire F6, Nouvelle session, taper \verb|carre| puis F6, Editeur script,
choisir Tortue ce qui cr\'ee un script contenant
l'instruction \verb|efface|. Ajouter 4 fois \verb|avance; tourne_gauche;|.
Appuyer sur EXE pour tester.
Quitter avec EXIT (et sauvegardez avec F1).

Taper EXIT. Modifiez avant la ligne \verb|efface| en
\begin{verbatim}
def f(n):
  for j in range(4):
    avance(n)
    tourne_gauche
\end{verbatim}
puis apr\`es la ligne \verb|efface;| tapez par exemple \verb|f(40)|
puis F6 1 Tester syntaxe ou EXE.

{\bf Un exemple de fonction non alg\'ebrique~: le calcul du PGCD de 2 entiers.}\\
Utiliser shift-EXE pour passer \`a la ligne. Le caract\`ere \verb|!|
est accessible depuis le menu \verb|Programmation_cmds| (F4 10, 
raccourci F4 0).
En syntaxe Xcas 
\begin{giacprog}
fonction pgcd(a,b)
  tantque b!=0 faire
    a,b:=b,irem(a,b);
  ftantque;
  return a;
ffonction
\end{giacprog}
Le m\^eme en syntaxe Python 
\begin{giacprog}
def pgcd(a,b):
  while b!=0:
    a,b=b,irem(a,b)
  return a
\end{giacprog}
On v\'erifie \giacinput{pgcd(12345,3425)}

{\bf Mise au point}\\
Depuis l'historique des calculs,
la commande \verb|debug| (disponible depuis shift-PRGM) permet 
d'ex\'ecuter une fonction en mode pas-\`a-pas, i.e.
visualiser l'\'evolution des variables instruction par instruction,
par exemple\\
\verb|debug(pgcd(12345,3425))|

\subsection{Quelques exemples}
Vous trouverez
\ahref{https://www-fourier.univ-grenoble-alpes.fr/\home{parisse}/casio/sessions}{en ligne} des exemples de sessions
dont certains sont des TP pour classes de seconde de l'IREM de Grenoble.

\subsection{Commandes utilisables}
Contrairement aux adaptations de MicroPython propos\'ees par
les constructeurs (dont
celui de la Casio Graph 90+e), la programmation en (simili-)Python 
dans KhiCAS n'est pas une application ind\'ependante. 
Vous pouvez donc utiliser tous les types
de Xcas (par exemple les rationnels) et appliquer toutes
les commandes de Xcas dans vos programmes. Ceci correspond
plus ou moins \`a un environnement Python avec les modules
\verb|math|, \verb|cmath|, \verb|random| (plus complet
que le module \verb|urandom| fourni par les constructeurs),
\verb|scipy|, \verb|numpy|, un petit module de graphiques
pixelis\'e 
(\verb|set_pixel(x,y,c)|, \verb|set_pixel()|
pour synchroniser l'affichage, \verb|clear()|,
\verb|draw_line(x1,y1,x2,y2,c)|, 
\verb|draw_polygon([[x1,y1],[x2,y2],...],c)|,
\verb|draw_rectangle(x,y,w,h,c)|, \verb|draw_circle(x,y,r,c)|,
la couleur+epaisseur+remplissage \verb|c| est un param\`etre optionnel,
\verb|draw_arc(x,y,rx,ry,t1,t2,c)| permet de tracer un arc
d'ellipse).
et pour remplacer \verb|matplotlib| on peut utiliser
les commande graphiques dans un rep\`ere de $\chi$CAS
(\verb|point|, \verb|line|, \verb|segment|, \verb|circle|,
\verb|barplot|, \verb|histogram| et les commandes \verb|plot...|).
De plus, vous pouvez travailler avec des expressions et faire
du calcul formel dessus.
La liste compl\`ete des commandes port\'ees se trouve en annexe,
pour une pr\'esentation d\'etaill\'ee, on renvoie \`a la documentation
de Xcas.

\section{Raccourcis claviers.} \label{sec:raccourcis}
\subsection{KhiCAS 90 et 50 (version en 2 fichiers)}
Ces raccourcis sont utilisables dans le shell et l'\'editeur texte
\begin{itemize}
  \item shift INS: table de caractères
\item F1 \`a F6, shift-F1 \`a shift F6: selon le bandeau en bas
\item OPTN: menu rapide des options de couleur
\item shift-OPTN: commandes de programmation
\item VARS: liste des variables
\item shift-PRGM: menu rapide utile en programmation
\item MENU: retour au menu principal Casio
\item shift-SETUP: configuration
\item EXIT: passage du shell \`a l'\'editeur et r\'eciproquement
\item shift-EXIT: affiche l'\'ecran de la tortue logo
\item ALPHA-EXIT: si le mode alpha n'est pas verrouill\'e, affiche le dernier graphe 2d ou 3d
\item shift angle: menu rapide g\'eom\'etrie 
\item fraction: menu rapide polyn\^omes dans le shell, indentation
  dans   l'\'editeur, force l'\'evaluation dans le tableur
\item shift-fraction: aide/compl\'etion
\item touche S$\leftrightarrow$D: applications additionnelles
  (tableur, finance, ...). Dans le tableur, affiche les graphes du tableur.
\item shift \verb|,|: \verb|;|
\item shift $\rightarrow$: \verb|:=| ou \verb|:| selon l'interpr\'eteur
  Xcas ou Python
\item AC/ON: annule la s\'election ou annule la recherche, sinon
efface la ligne courante et la copie dans le presse-papier
\item shift CAPTURE: sauve la session ou le fichier texte courant
\item shift CLIP: d\'ebut de s\'election de zone ou copie de
  s\'election dans le presse-papier
\item shift PASTE: colle le presse-papier
\item shift CATALOG: liste toutes les commandes Xcas
\item shift FORMAT: commandes de programmation
\item shift 6: menu rapide avec \verb|<>_!| et \verb|comb, rand, binomial,  normald|
\item shift List: menu rapide liste
\item shift Mat: menu rapide matrice
\item shift 3: menu rapide alg\`ebre (bi-)lin\'eaire
\item shift EXE: passe \`a la ligne dans l'\'editeur
\end{itemize}
Les raccourcis claviers ouvrant un menu rapide sont configurables, en
\'editant le fichier \verb|FMENU.py|. Pour revenir \`a la
configuration par d\'efaut, effacez ce fichier depuis l'application
Memory de la Casio.

\subsection{KhiCAS (version courte en 1 fichier)}
\begin{itemize}
\item F1 \`a F3~: selon  les l\'egendes
\item F4: catalogue de commandes.
\item F5: passage majuscules minusucules. Si le mode alphab\'etique
n'est pas actif, bloque le clavier en mode alpha minuscule.
\item F6: menu Fichier et configuration.
\item \verb|(-)|: dans l'\'editeur de programmes,
renvoie le caract\`ere \verb|_|. 
\item shift PRGM: commandes de programmation
\item OPTN: toutes les options 
\item shift-QUIT: commandes tortues
\item shift-List: commandes et \'edition de listes
\item shift-Mat: commandes et \'edition de matrices
\item touche angle: commande nombres complexes
\item touche fraction: indentation dans l'\'editeur, menu
divers dans l'historique
\item touche fraction shift\'ee (jaune): graphes
\item touche r rouge: \verb|abs|
\item touche $\theta$ rouge: \verb|arg|
\item touche S$\leftrightarrow$D: commandes sur les nombres r\'eels
\item touche S$\leftrightarrow$D shift\'ee (jaune): commandes sur les entiers
\item Shift-INS (touche DEL): \verb|:=|
\item Depuis le shell, taper sur la touche fl\`eche vers le bas 
pour obtenir de l'aide ou/et compl\'etion.
Depuis l'\'editeur de programmes, taper sur shift touche de fraction (G).
\end{itemize}

Dans l'\'editeur de programmes~:
\begin{itemize}
\item touches curseur shift\'ees: d\'eplacement en d\'ebut/fin de ligne/fichier.
\item shift CLIP: s\'election. D\'eplacer le curseur \`a l'autre
extr\'emit\'e puis taper sur DEL pour effacer la s\'election
et la copier dans le presse-papier ou \`a nouveau sur
shift CLIP pour copier la s\'election sans l'effacer.
Taper sur AC/ON pour annuler.
\item EXE: si une recherche/remplacement
de mot est active (apr\`es avoir fait F6 6),
recherche l'occurence suivante d'un mot. Sinon, teste la syntaxe.
\item shift EXE: passe \`a la ligne
\item DEL efface le caract\`ere pr\'ec\'edent ou la s\'election.
\item shift PASTE: copie le presse-papier
\item AC/ON: annule la s\'election ou annule la recherche, sinon
efface la ligne courante et la copie dans le presse-papier
%\item VARS: bascule entre l'\'editeur et la figure tortue
\item EXIT: quitte l'\'editeur et revient en ligne de commandes. On
peut revenir ensuite \`a l'\'editeur en tapant \`a nouveau EXIT.
\end{itemize}

\section{Remarques}
Pour \'eteindre la Graph 90 depuis KhiCAS, taper d'abord 
sur la touche MENU puis shift ON. Lorsque vous rallumez la
calculatrice, taper sur MENU.

Pour connecter la Graph 90 au PC comme une clef USB lorsqu'on
est dans KhiCAS, il faut 
taper sur une touche  apr\`es avoir tap\'e sur F1, sinon
rien ne se passe.

Si $\chi$CAS se bloque dans un long calcul, essayez de taper sur la
touche AC/ON. Si cela n'a pas d'effet, vous pouvez toujours
r\'einitialiser la calculatrice avec le bouton reset \`a l'arri\`ere.

Si $\chi$CAS plante avec un affichage du type SYSTEM ERROR, etc.
essayez de taper sur la touche MENU puis de lancer une autre application,
si tout se passe bien, vous pourrez sauvegarder votre session et
relancer $\chi$CAS sans avoir \`a effectuer une r\'einitialisation
compl\`ete. En cas de probl\`emes persistants, 
depuis le MENU principal, ouvrir l'application M\'emoire, puis F2
(m\'emoire de stockage) puis effacez le fichier \verb|session.xw|
(d\'eplacez le curseur sur le fichier, puis F1 pour SELECT puis
F6 pour DELETE). Si $\chi$CAS90 se lance (la version en deux fichiers), 
vous pouvez aussi  depuis le menu F6 s\'electionner le dernier item 
Quit et choisir la r\'einitialisation de KhiCAS.

Vous disposez dor\'enavant d'environ 500K pour vos donn\'ees sur une Graph 90
(ce qui permet de travailler assez confortablement)
et de 58K sur une Graph 35eii (ce qui est vite rempli).
Sur la Graph 90, il est conseill\'e d'ouvrir le menu VARS de temps
en temps, si vous constatez que la m\'emoire disponible descend en-dessous
des 100K, relancez $\chi$CAS (touche MENU, puis ouvrir n'importe
quelle autre application puis MENU et r\'eouvrir $\chi$CAS).

\section{Version plus compl\`ete (en 2 fichiers)} \label{sec:2}
Cette version de $\chi$CAS en deux fichiers contient un portage de
MicroPython avec plusieurs modules utiles pour l'enseignement, plus de 
commandes de Xcas, en particulier des commandes de g\'eom\'etrie,
d'alg\`ebre avanc\'ee, le support de flottants avec grande
pr\'ecision, un moteur d'affichage 3d. Le menu Fichier permet d'acc\'eder
\`a des applications additionnelles : tableur formel, géométrie,
finance, table p\'eriodique des \'el\'ements ...

Pour installer cette version plus compl\`ete, cf. la section
\ref{sec:install}. 

\subsection{MicroPython 1.12}
Pour passer du shell de Xcas \`a celui de MicroPython, tapez
shift-SETUP et s\'electionnez MicroPython. 
Lorsque MicroPython est actif, les menus sont sur fonds
jaune-brun, sinon sur fonds rose (Xcas) ou bleu (Xcas compatible Python).
Les modules disponibles sont: turtle (version plus compl\`ete que
celle de Casio, avec en particulier la possibilit\'e de remplir des formes),
graphic (version plus compl\`ete de casioplot, avec compatibilit\'e
Numworks), matplotl (graphes rep\'er\'es), 
arit (arithm\'etique enti\`ere), linalg/numpy (alg\`ebre lin\'eaire),
ulab (compatibilit\'e scipy), cas (calcul formel dans Python).

\subsection{Graphes 3d et 4d}
Cette version de $\chi$CAS en deux fichiers contient 
toutes les commandes de base de Xcas de g\'eom\'etrie 2d et 3d. 
On peut en particulier tracer des graphes de fonction de 2 variables,
des cones, des plans, etc. Par exemple tapez 
\verb|F4 * 5 F2| pour s\'electionner un exemple de cube
ou \verb|shift F4 1| pour s\'electionner la commande plot,
puis tapez \verb|x^2-y^2|.

Pour tracer rapidement le graphe d'une fonction de $\mathbb{C}$ dans
$\mathbb{C}$, on peut utiliser la commande \verb|plot| avec
en argument une fonction de 2 variables (partie réelle
et imaginaire) à valeurs complexes,
par exemple \verb|plot((x+i*y)^2-9)| ou pour les fonctions
holomorphes/méromorphes ayant une expression en fonction de la variable
complexe sans passer par partie réelle et imaginaire,
on peut utiliser la commande \verb|plot3d| par exemple
\verb|plot3d(x^2-9)| (la commande \verb|plot(x^2-9)| ne
convient pas car déjà utilisée pour un graphe de $\mathbb{R}$ dans
$\mathbb{R}$. Le module de la fonction est représenté selon $z$,
l'argument est représenté par les couleurs de l'arc en ciel, de $-\pi$
en bleu violet à 0 en vert (en passant par jaune et orange) et de 0 à $\pi$
en passant par cyan.

Pour préciser des options, il faut utiliser la commande \verb|plotfunc|,
par exemple\\
\verb|plotfunc((x+i*y)^3-1,[x=-2..2,y=-2..2],nstep=500)|\\
pour tracer $z \rightarrow z^3-1$ depuis le carré du plan complexe
centré en l'origine
de coté 4, avec une discrétisation utilisant 500 petits rectangles.

Le moteur de rendu 3d \'etant relativement lent sur la calculatrice, la
pr\'ecision de trac\'e par d\'efaut est moyenne ce qui impacte
en premier lieu le rendu des objets avec des angles, comme les poly\`edres. 
On peut changer
la pr\'ecision du trac\'e en tapant F2 (plus rapide, moins pr\'ecise) 
ou F3 (moins rapide, plus pr\'ecis). Notez qu'il est possible
d'accélérer la vitesse du processeur 
au lancement de $\chi$CAS (utilise du code de
S\'ebastien Michelland, semblable \`a l'addin
\ahref{https://www.planet-casio.com/Fr/programmes/dl.php?id=3710&num=1}{Ptune3})

Pour obtenir un unique trac\'e pr\'ecis sans changer la pr\'ecision
par d\'efaut, tapez sur la touche \verb|^| et
soyez patients. On peut interrompre un trac\'e pr\'ecis en cours en tapant
sur la touche DEL.
Les touches de d\'eplacement permettent de changer de point de vue, 
taper 5 pour revenir au point de vue initial,
\verb|-| ou \verb|+| pour faire un zoom in ou out. Lorsqu'on laisse
une touche de d\'eplacement appuy\'ee, la pr\'ecision de trac\'e est
diminu\'ee temporairement pour donner une certaine fluidit\'e,
lorsqu'on rel\^ache la touche, la derni\`ere position est retrac\'ee
avec la pr\'ecision par d\'efaut. 
F4 permet de montrer ou cacher une 2\`eme surface cach\'ee, F5
affiche ou cache des points interm\'ediaires, F6 montre ou cache
les ar\^etes des poly\`edres.

\subsection{G\'eom\'etrie interactive 2d/3d}
L'application de géométrie
permet de construire des figures dans le plan ou dans l'espace,
et de faire bouger un point et tout ce qui en d\'epend pour illustrer
certaines propriétés (géométrie
dynamique). On peut faire des constructions de géométrie euclidienne
pure, mais aussi avec des graphes de fonction, des coniques, etc.
L'application possède deux ``vues'': la vue graphique et la vue
symbolique qui contient les commandes Xcas permettant de créer la figure
(la philosophie de cette application est proche de celle du logiciel
Geogebra, avec les commandes de Xcas).

Cette section donne un aper\c{c}u rapide de l'application,
pour une documentation plus compl\`ete avec quelques captures d'\'ecran, cf.
\ahref{https://www-fourier.univ-grenoble-alpes.fr/\home{parisse}/casio/geo.html}{ici}

\subsubsection{Modes, vue graphique et symbolique.}
Taper F6 1 (ou la touche S$\leftrightarrow$D) pour afficher la liste des
applications additionnelles, puis EXE puis sélectionnez soit une nouvelle
figure 2d ou 3d soit une figure existante. Vous pouvez aussi ouvrir
l'application de géométrie depuis un graphe (par exemple après avoir
tapé \verb|plot(sin(x))|) en tapant F6 puis Sauvegarder figure.

Au lancement on est dans la vue graphique en mode repère, les touches de curseur
permettent de changer de point de vue. Pour changer le mode, utiliser
la touche F4, pour passer en vue symbolique et vice-versa taper EXE.
Par exemple tapez F4 3 pour passer en mode point
qui permet de construire des points en déplaçant
le pointeur et en tapant EXE ou tapez F4 5 pour passer
en mode triangle qui permet de construire
un triangle à partir de ses 3 sommets, on déplace le pointeur et on tape EXE
trois fois. Pour déplacer le pointeur, utiliser les touches de déplacement,
pour se déplacer plus rapidement, faire shift touche de curseur. Si on
est proche d'un point existant, son nom apparait en bas. Pour
déplacer le pointeur vers un point existant, vous pouvez aussi
taper le nom du point (par exemple touche alpha A).
Pour une figure 3d, les objets créé depuis la vue graphique seront situés
dans le plan qui apparait en jaune sur la figure. On peut modifier ce plan
en utilisant les touches 4 et 6. Il est fortement conseillé de
conserver un point de vue avec l'axe $Oz$ vertical (donc changer
de point de vue uniquement avec les touches flèche droite ou gauche
qui effectuent une rotation autour de $Oz$).

Le mode pointeur permet de sélectionner un point et de le
déplacer pour observer comment la construction varie, ce qui permet
de mettre en évidence des propriétés de la figure, par exemple concurrence
de 3 droites.

Si vous tapez sur la touche EXIT depuis la vue graphique
de l'application, vous revenez au mode repère ou si vous y étiez
vous passez en vue symbolique. Vous pouvez ajouter
des objets à la construction depuis cette vue, en mettant une commande
par ligne. Tapez shift-EXE pour passer à la ligne. Tapez EXE
pour revenir à la vue graphique. Dans la vue symbolique, vous
pouvez sauvegarder la construction géométrique au format texte
(avec une extension \verb|.py|, même s'il ne s'agit pas
d'un script Python).
Tapez EXIT pour quitter l'application de géométrie.

Lorsque vous quittez l'application de géométrie,
la figure est automatiquement sauvegardée
dans une variable Xcas qui a le même nom que celui du nom de fichier
affiché dans la vue symbolique. Vous pouvez purger la variable Xcas
si vous voulez effacer la figure de la session.

{\bf Exemple~: cercle circonscrit.}\\
Depuis le shell, taper F6 1 sélectionner nouvelle figure 2d et valider EXE.
Puis F4 5 Triangle, EXE pour créer le premier sommet du triangle
puis déplacer le pointeur avec les touches de déplacement, EXE pour créer
le 2ème sommet du triangle, déplacer le pointeur à nouveau et EXE
pour créer le triangle.

Version longue en construisant le centre~:
Taper F4 7, sélectionner 8 Mediatrice, déplacer
le pointeur de sorte que seul un segment du triangle
soit sélectionné (affichage
en bas à droite \verb|perpen_bisector D5,D|),
taper EXE pour créer la médiatrice du segment, déplacer le pointeur
sur une autre arête du triangle et EXE pour créer la 2ème médiatrice,
optionnellement sur le 3ème segment pour avoir les 3 médiatrices.
Puis F4 6 et 4 Intersection unique. Déplacer le pointeur vers
une des médiatrices, taper EXE puis vers une autre médiatrice, taper EXE,
ceci crée le centre du cercle circonscrit. Pour tracer le cercle,
taper F4 4, déplacer le pointeur au centre du cercle (vous pouvez
utiliser les touches de déplacement ou juste taper
ALPHA H ou la bonne lettre si le centre du cercle s'appelle autrement),
puis EXE puis sur un des sommets et EXE.

Version courte avec la commande \verb|circonscrit|~:
taper F4 9 puis circonscrit puis sélectionner chaque
sommet avec EXE (ALPHA A EXE ALPHA B EXE ALPHA C EXE, remplacez A, B, C
par les lettres du sommet du triangle).

Version en vue symbolique~:
taper EXIT puis en fin de script sur une ligne vide (taper shift EXE s'il
faut en créer une), taper\\
\verb|c:=circonscrit(A,B,C)| EXE

{\bf Exemple 3d: bac septembre 2019~}\\
Taper F6 1 pour lancer l'application de géométrie puis nouvelle figure 3d.
Puis EXIT ou EXE pour passer en vue symbolique. Puis F5 c ALPHA shift =
puis F4 flèche haut deux fois pour sélectionner 3D puis EXE puis 5 pour cube
puis F6 (aide), qui explique que
les 2 premiers arguments de cube sont les sommets d'une ar\^ete,
le troisième est un point d'un plan d'une face. Le premier
exemple nous convient ici exactement, on tape F2 et on obtient
\verb|c=cube([0,0,0],[1,0,0],[0,1,0])|
On tape EXE pour voir le cube puis + plusieurs fois pour zoomer et EXE
pour revenir à la vue symbolique.
Vous pouvez sauvegarder à tout moment la construction au format
texte depuis
le menu F6.
On passe à la ligne en tapant shift EXE. Puis on définit les sommets
du cube en tapant \verb|A,B,C,D,E,F,G,H=|
(taper ALPHA A , ALPHA B etc.), puis F4 et flèche vers le haut 3 fois
pour sélectionner Géometrie puis flèche vers le haut 4 fois pour sélectionner
\verb|sommets| EXE et mettre c en argyument \verb|sommets(c)|.
Taper EXE pour visualiser puis EXE à nouveau pour revenir en vue
symbolique. Passer à la ligne avec shift EXE puis créer le plan ABG
en tapant ALPHA P = puis F2 pour
ouvrir le menu rapide lines et 8 pour saisir plane. La commande plan
prend en arguments 3 points pour définir le plan (on peut aussi donner
une équation cartésienne, ici A,B,G, \verb|P=plan(A,B,G,|
on va lui ajouter une couleur avec le menu rapide F3 disp
\verb|display=filled+green|, vérifier en visualisant avec EXE EXE.
On passe à la ligne (shift EXE) et on crée le segment DE
ALPHA S = F2 sélectionner la commande segment avec EXE
puis D,E et F3 pour lui donner une couleur
\verb|S=segment(D,E,color=cyan)|
(on pouvait aussi créer le segment depuis la vue graphique en mode Lignes
mais sur la Casio c'est un peu lent).
La construction est donc la suivante:
\begin{verbatim}
c=cube([0,0,0],[1,0,0],[0,1,0])
A,B,C,D,E,F,G,H=sommets(c)
P=plan(A,B,G,display=filled+green)
S=segment(D,E,display=cyan)
\end{verbatim}
Vous pouvez taper EXE pour la visualiser et utiliser les flèches de
déplacement pour changer de point de vue. Taper EXE ou EXIT pour revenir
en vue symbolique. Pour quitter l'application taper EXIT. Taper F1 pour
sauvegarder la figure si nécessaire.
Vous pouvez depuis le shell de KhiCAS accéder à de nombreuses informations
de géométrie analytique, par exemple \verb|equation(P)| (menu
F4 Géométrie) vous donnera
l'équation cartésienne du plan $P$ ou \verb|is_orthogonal(P,S)|
(F4 Géométrie) vous confirmera que le plan $P$ est orthogonal au segment $S$.


%% {\bf Exemple 3d~: intersection d'un cône avec un plan}\\
%% Taper F6 1 pour lancer l'application de géométrie puis nouvelle figure 3d.
%% Puis EXIT ou EXE pour passer en vue symbolique. Saisir ALPHA C shift =
%% puis F4 flèche haut deux fois pour sélectionner 3D puis EXE, puis 3 (cone) F2 (exemple 1), ceci affiche la commande pour générer un cône de centre l'origine
%% et d'axe vertical, d'angle $\pi/6$. Taper EXE pour revenir en vue
%% graphique. Après l'affichage du cone, taper F4 et flèche vers le haut trois fois
%% pour sélectionner Curseur, EXE, ce qui repasse en vue symbolique et génère
%% une commande de type \verb|a:=element(0..1,0.5)|, modifiez-la
%% en \verb|a:=element(-4..4,-2)|. Taper
%% alors sur une nouvelle ligne la commande
%% \verb|P:=plan(z=a*y+3,display=filled+cyan)|
%% en utilisant F5 plan puis shift-F3 pour taper display, filled et cyan.
%% Ceci crée un plan penché dont on peut modifier l'inclinaison
%% en changeant le curseur.


\subsubsection{Curseurs}
Vous pouvez créer un paramètre qui se déplace entre 2 valeurs extrêmes
par saut de 1\% depuis le menu F4 Curseur en vue graphique ou
la commande \verb|element| en vue symbolique.

{\bf Exemple~: explorateur quadratique avec des curseurs}\\
pour explorer comment une parabole d'équation $y=ax^2+bx+c$
dépend de la valeur de $a,b,c$, créer 3 curseurs
(F4 curseur vers le haut 4 fois EXE EXE). En vue symbolique
on doit avoir quelque chose ressemblant à
\begin{verbatim}
a:=element(-1..1)
b:=element(0..1,0.5)
c:=element(-1..1)
\end{verbatim}
puis ajouter un graphe, depuis la vue graphique taper F4 0 (pour 10 Courbe)
et sélectionner plot,
ou en vue symbolique shift F6 et sélectionner plot,
remplir entre les parenthèses par \verb|a*x^2+b*x+c| (attention
à ne pas oublier les \verb|*|), puis valider avec EXE. En vue graphique,
vous devez voir 3 curseurs \verb|a|, \verb|b| et \verb|c| et le
graphe correspondant. Vous pouvez maintenant faire varier les valeurs
de $a,b,c$ depuis la vue graphique en mode pointeur (F4 2) en déplaçant le
pointeur vers $a,b,c$ et en tapant EXE puis les flèches de déplacement
gauche ou droit et EXE pour arrêter.

On peut bien sur faire des exemples plus simples avec un ou deux curseurs
et une courbe dépendant d'un ou de deux paramètres. Par exemple
un explorateur linéaire \verb|droite(y=a*x+b)| ou
trigonométrique \verb|plot(sin(a*x+b))|.

\subsubsection{Mesures et légendes}
En tapant F4 puis en sélectionnant 13 Mesures,
on peut afficher une mesure en un point de la figure. Par exemple
après avoir construit un triangle, on peut afficher son périmètre
ou son aire en tapant F4 puis deux fois flèche vers le haut, EXE.
Déplacer le pointeur près du triangle, taper EXE, puis déplacer
le curseur à l'endroit où on souhaite mettre la mesure et taper EXE.

On peut afficher n'importe quelle légende avec la commande
\verb|legende()| depuis la vue symbolique. Le premier argument
de légende peut être un point de la figure, ou bien un vecteur
de deux entiers donnant la position absolue en pixels mesuré
depuis le coin en haut à gauche. Le deuxième argument est la légende,
cela peut être une chaine de caractères ou n'importe quelle expression.

Si la légende est une valeur numérique, elle peut être utilisée pour
un paramètre numérique d'une commande, par exemple le rapport d'une
homothétie ou l'angle d'une rotation en utilisant la commande
\verb|extract_measure|. Par exemple
\verb|r:=legend([20,40],"2")|\\
\verb|homothetie(A,extract_measure(r),B)| 

\subsubsection{Traces}
La commande \verb|trace()| permet de conserver la trace du
déplacement d'un objet géométrique.

{\bf Exemple} Enveloppe des normales à une courbe paramétrée
(ici une ellipse).\\
\begin{verbatim}
E:=plotparam([cos(t),2*sin(t)],t=-pi..pi)
a:=element(-pi..pi)
M:=element(E,a)
T:=tangent(M)
N:=perpendiculaire(M,T)
trace(N)
\end{verbatim}
En faisant varier $a$,
on observe une courbe séparant la zone des normales à une zone sans point
tracé qui est l'enveloppe des normales, c'est la développée de l'ellipse\\
\verb|evolute(E,color=red)|\\
On peut effacer les traces à tout moment
avec la touche F6, puis touche vers le haut pour sélectionner
le dernier item du menu de configuration.

\subsection{Tableur formel}
Le tableur de $\chi$CAS est un tableur utilisant la syntaxe standard des
tableurs pour les références de cellule, mais il
est capable de travailler avec des donn\'ees exactes
(par exemple fraction ou racines carr\'ees) et symboliques (par
exemple on peut calculer la d\'eriv\'ee d'une cellule du tableur),
et on dispose de l'ensemble des commandes de Xcas, ainsi que les
fonctions définies par l'utilisateur.

Taper F6 1 (ou la touche S$\leftrightarrow$D) pour afficher la liste des
applications additionnelles, s\'electionner le tableur et valider avec
EXE. 

Pour donner une valeur fixe \`a une cellule, tapez simplement
un nombre. Vous pouvez aussi taper une chaine de caract\`ere
ou un autre objet reconnu par Xcas. Si vous saisissez une
liste ou une commande renvoyant une liste, celle-ci remplira
plusieurs cellules (en allant vers le bas ou vers la droite selon
la configuration du tableur). Par exemple \verb|range(10)|
remplira 10 cellules cons\'ecutives avec les entiers de 0 \`a 9
(pour saisir \verb|range|, vous pouvez utiliser le menu rapide F1).

Pour d\'efinir le contenu d'une cellule en faisant r\'ef\'erence
\`a d'autres cellules, taper \verb|=|
puis utilisez la syntaxe
habituelle des tableurs,  les
caract\`eres \verb|:| et \verb|$| sont accessibles depuis
le menu F3 edit, \verb|:| est aussi accessible par shift
$\rightarrow$. 
On peut s\'electionner une cellule lorsqu'on \'edite
le contenu d'une autre cellule en d\'epla\c{c}ant le curseur vers
le haut ou vers le bas, puis vers n'importe quelle direction.
Pour s\'electionner une plage de cellules, d\'eplacez-vous
vers un des coins de la plage, taper shift-CLIP, puis
d\'eplacez-vous vers l'autre coin et tapez EXE. 

Dans la d\'efinition d'une cellule ,
vous pouvez utiliser toutes les commandes de Xcas 
via le menu F4
ou les menus rapides (F1 \`a F6 \'eventuellement pr\'ec\'ed\'es
de alpha ou shift) ou shift CATALOG. Vous pouvez aussi utiliser les structures
de controle Xcas (test/boucle) ainsi que des fonctions
Xcas que vous avez d\'efinis. Attention, seules les fonctions
d\'efinies dans l'interpr\'eteur Xcas sont reconnus (quelle que soit
la syntaxe utilis\'ee), les programmes
\'ecrits en MicroPython ne sont pas reconnus.


Une cellule peut contenir une commande renvoyant un r\'esultat 
graphique. Pour voir le graphique correspondant \`a toutes les cellules du tableur ayant
un r\'esultat graphique, taper la touche S$\leftrightarrow$D.

\section{Copyright et Remerciements}
\begin{itemize}
\item Giac et $\chi$CAS, noyau de calcul (c) B. Parisse et R. De Graeve, 2019.
\item Interface de $\chi$CAS adapt\'ee par B. Parisse \`a partir
de l'interface utilisateur du code source d'Eigenmath
cr\'e\'ee par Gabriel Maia pour la 90 et par Mike Smith et Nemh pour
la 35eii, et de l'interface utilisateur de Xcas.
\item
License d'utilisation de $\chi$CAS: GPL2. Voir le d\'etail
des conditions dans le fichier \verb|LICENSE.GPL2| de l'archive
\ahref{https://www-fourier.univ-grenoble-alpes.fr/\home{parisse}/casio/khicasio.zip}{khicasio.zip}
ou sur la page
\ahref{https://www.gnu.org/licenses/old-licenses/gpl-2.0.fr.html}{GPL2}
du site de la Free Software Foundation.
Le code source de $\chi$CAS ainsi que des librairies 
libtommath et USTL, se trouvent sur
\ahref{https://www-fourier.univ-grenoble-alpes.fr/\home{parisse}/casio/}{la section
Casio} de ma page web (cf. la section \ref{sec:dev})
\item Certaines parties du code sont sous licence MIT: 
\begin{itemize}
\item \ahref{https://github.com/micropython/micropython/blob/master/LICENSE}
{MicroPython 1.12 (c) Damien P. George}
\item \ahref{https://www.nayuki.io/page/qr-code-generator-library}
{QR code generator, project Nayuki}
\end{itemize}
\item Remerciements \`a tous les membres actifs de tiplanet et
de Plan\`ete Casio pour l'aide qu'ils m'ont apport\'es
en r\'epondant \`a mes questions ou/et en testant $\chi$CAS,
en particulier \`a LePhenixNoir pour son expertise 
et son aide technique pour la 35 et la 90, Redoste pour le
debugger, Nemhardy pour la 90, et \`a critor
pour les articles, tests et la diffusion de KhiCAS via tiplanet.
Remerciements \`a tous les contributeurs du
\ahref{http://prizm.cemetech.net/index.php/Prizm_Programming_Portal#Native_OS_reference}{Prizm programming portal}.
Remerciements \`a Pavel Demin pour les astuces de compilation ayant
permis d'\'economiser 135K environ.
\item Remerciements \`a Camille Margot pour l'int\'er\^et
port\'e au projet et \`a Casio France pour le soutien par le pr\^et de
calculatrices Graph 90/35 et l'attribution 
d'une licence d'utilisation de l'\'emulateur.
\end{itemize}

\section{Réinstallation de l'OS d'origine} \label{sec:oserror}
Téléchargez un installeur, par exemple
\ahref{https://tiplanet.org/forum/archives_list.php?order=hit&cat=OS+cprizm&multi_chaine_search=3.60}{ici}.
Lancez le programmes
d'installation de Casio sur le PC jusqu'à la dernière confirmation,
mais au lieu de taper F5 OS Update sur la calculatrice,
effectuez les manipulations suivantes (merci à critor, TI-Planet)~:
\begin{enumerate}
     \item maintenir enfoncé le bouton restart    au dos
    sans le relâcher, maintenir enfoncées les touches
    F2,   4 et    AC
\item
  sans rien relâcher d'autre, relâcher le bouton restart
  \item
    maintenant relâcher uniquement les touches
    F2 et    4
\item
  patienter une seconde
  \item
    maintenant relâcher enfin la touche    AC
\item
    maintenir enfoncée la touche    9,
    patienter une seconde
    relâcher la touche    9
\item
    maintenir enfoncée la touche *
\item
  La calculatrice doit se rallumer seulement maintenant
  sur un écran "OS ERROR",  relâcher la touche *
\item vous pouvez maintenant exécuter la dernière confirmation
  d'installation de l'OS sur l'ordinateur.
\end{enumerate}
N.B.: la manipulation ci-dessus est assez acrobatique, il faudra probablement
faire plusieurs essais avant d'avoir l'écran OS Error.
Il peut aussi être nécessaire de laisser reposer la calculatrice (éteinte)
environ 1h avant.
Cette manipulation fonctionne aussi sur les Math+, mais
{\bf pour l'instant $\chi$CAS n'est
pas compatible avec le mode examen sur les Graph Math+.}.

\section{\`A propos du d\'eveloppement.} \label{sec:dev}
\subsection{Install rapide sous linux} 
Installer
\ahref{https://www-fourier.univ-grenoble-alpes.fr/\home{parisse}/casio/libmpfr.so.4}{la
librairie dynamique libmpfr.so.4} dans \verb|/usr/local/lib|,
v\'erifiez que \verb|/usr/local/lib| est dans la liste des chemins de
\verb|/etc/ld.so.conf| 
et ex\'ecutez
\verb|sudo ldconfig|. Puis d\'esarchivez
\ahref{https://www-fourier.univ-grenoble-alpes.fr/\home{parisse}/casio/casiolocal.tgz}{casiolocal.tgz},
c'est une version du cross-compiler gcc avec certaines librairies
(libc, ustl, tommath). Pour cr\'eer des addins pour la 90, vous devez aussi installer 
\ahref{https://www-fourier.univ-grenoble-alpes.fr/~parisse/casio/mkg3a.tgz}{mkg3a}.

\subsection{Source de giac}
Le source pour la 90 est 
%\ahref{https://www-fourier.univ-grenoble-alpes.fr/\home{parisse}/casio/giac2.tgz}{giac2.tgz}
% ou
\ahref{https://www-fourier.univ-grenoble-alpes.fr/\home{parisse}/casio/giacbf.tgz}{giacbf.tgz}
(avec support des flottants multi-pr\'ecision). Il devrait se compiler avec
\verb|make|.
La version light en un seul fichier est 
\ahref{https://www-fourier.univ-grenoble-alpes.fr/\home{parisse}/casio/giac90.tgz}{giac90.tgz}.
Pour la 35gii, c'est
\ahref{https://www-fourier.univ-grenoble-alpes.fr/\home{parisse}/casio/giac35.tar.bz2}{giac35.tar.bz2}
et on fait \verb|make| dans le r\'epertoire \verb|giac35/src0|.

Si quelque chose ne fonctionne pas, voici quelques d\'etails.
Pour d\'evelopper l'add-in, j'ai install\'e le cross-compiler
gcc pour processeur sh3eb, en m'inspirant de ce
\ahref{https://www.planet-casio.com/Fr/forums/topic12970-1-Tutoriel_Compiler_sous_Linux_avec_un_cross-compilateur_gcc.html}{tutoriel}.
T\'el\'echargez gcc puis 
(remplacer la version de gcc par la version t\'el\'echarg\'ee)\\
\verb|../gcc-5.3.0/configure --target=sh3eb-elf --prefix="$HOME/opt/sh3eb-elf" --disable-nls --disable-shared --disable-multilib --enable-languages=c,c++ --without-headers|\\
Malheureusement, il n'y a pas de support pour sh3eb dans la \verb|newlib|
(librairie C) fournie avec gcc, encore moins pour la \verb|libstdc++|.

J'ai donc install\'e \verb|libfxcg.tar.gz|, 
quelque peu modifi\'ee par mes soins 
(corrections de plusieurs petits bugs dans la librairie C, ajout
de fonctions manquantes comme qsort, ...), \`a r\'ecup\'erer dans
\ahref{https://www-fourier.univ-grenoble-alpes.fr/\home{parisse}/casio/}
{ce r\'epertoire} (d\'esarchiver et compiler avec make). Toujours
dans ce r\'epertoire, on trouve \verb|tommath.tgz| (gestion des
entiers multi-pr\'ecision) et \verb|ustl.tar.gz| (impl\'ementation
de la standard template library) que j'ai du pas mal modifier
pour le faire fonctionner avec sh3eb-elf-g++, avec un r\'esultat
partiel qui est suffisant pour porter Giac (support de vector/string/map
mais pas des flux E/S fichiers, j'ai cr\'e\'e \`a part un fichier
iostream pour avoir un ersatz de cin/cout, cf. le r\'epertoire 
\ahref{https://www-fourier.univ-grenoble-alpes.fr/\home{parisse}/casio/ustl}{ustl}). 
Tout ces fichiers se d\'esarchivent
et on compile en principe avec make.

Comme il n'y
a pas de support du compilateur pour la newlib et la libstdc++ et pour
\'eviter les probl\`emes d'ordre d'initialisation des variables statiques,
ce portage de Giac se caract\'erise par l'absence de variables statiques
classes, il n'y a que des variables statiques C. La tr\`es grande
majorit\'e des variables
globales (tous les noms de commandes Xcas par exemple) sont d\'eclar\'ees via
des alias sur des structures C en constantes pour ne
pas occuper de place en m\'emoire vive. Il a fallu tenir compte
de l'endianness du CPU par rapport \`a d'autres portages de Giac
(plus pr\'ecis\'ement le CPU peut se comporter des deux mani\`eres
little ou big endian mais on est oblig\'e de choisir le m\^eme
ordre que celui de l'OS Casio, et c'est l'inverse de celui qu'on
rencontre sur les architectures Intel).

\subsection{Mise au point}
Les etapes pour utiliser gdb pour Casio sous linux avec wine sont les suivantes:
\begin{enumerate}
\item installer wine si n\'ecessaire 
\item installer l'emulateur casio avec wine avec une commande qui
ressemble \`a\\
\verb|wine /chemin_vers/fx-CG_Manager_PLUS_Subscription_for_fx-CG50series_Ver.3.40.exe|
\item
installer l'equivalent de gdb-server
\begin{verbatim}
cd .wine/drive_c/Program\ Files\ \(x86\)/CASIO/fx-CG\ Manager\ PLUS\ Subscription\ for\ fx-CG50series/
mv CPU73050.dll CPU73050.real.dll
wget https://www-fourier.univ-grenoble-alpes.fr/~parisse/casio/CPU73050.dll
cp CPU73050.dll CPU73050.dbg.dll
\end{verbatim}
(j'ai mis la dll sur ma page, parce que j'ai eu un peu de mal a le
compiler, le source est 
\verb|https://github.com/redoste/fx-CG50_Manager_PLUS-gdbserver|).

\item compiler gdb pour sh3:
\begin{verbatim}
wget https://ftp.gnu.org/gnu/gdb/gdb-11.1.tar.gz
cd casio (ou autre repertoire de build)
tar xvfz ../gdb-11.1.tar.gz
mkdir sh3eb-gdb
cd sh3eb-gdb
../gdb-11.1/configure --srcdir=../gdb-11.1 --target=sh3eb-elf
\end{verbatim}
(ajouter --prefix=chemin dans la commande configure si on n'a pas les
droits sur /usr/local/bin).
Puis
\begin{verbatim}
make
sudo make install
\end{verbatim}

Ensuite on cr\'ee deux scripts, \verb|casioemu| pour lancer
en mode normal
\begin{verbatim}
#! /bin/bash
cd ~/.wine/drive_c/Program\ Files\ \(x86\)/CASIO/fx-CG\ Manager\ PLUS\ Subscription\ for\ fx-CG50series
/bin/cp CPU73050.real.dll CPU73050.dll
cd
wine "C:\Program Files (x86)\CASIO\fx-CG Manager PLUS Subscription for fx-CG50series\fx-CG_Manager_PLUS_Subscription_for_fx-CG50series.exe" /n"fx-CG Manager PLUS Subscription for fx-CG50series" > /dev/null &
\end{verbatim}
et \verb|casiodbg| pour lancer en mode debug
\begin{verbatim}
#! /bin/bash
cd ~/.wine/drive_c/Program\ Files\ \(x86\)/CASIO/fx-CG\ Manager\ PLUS\ Subscription\ for\ fx-CG50series
/bin/cp CPU73050.dbg.dll CPU73050.dll
cd
wine "C:\Program Files (x86)\CASIO\fx-CG Manager PLUS Subscription for fx-CG50series\fx-CG_Manager_PLUS_Subscription_for_fx-CG50series.exe" /n"fx-CG Manager PLUS Subscription for fx-CG50series" > /dev/null &
\end{verbatim}

\item Pour utiliser l'\'emulateur en mode normal 
taper simplement \verb|casioemu|, pour le mode debug taper
\begin{verbatim}
casiodbg
sh3eb-elf-gdb
target remote localhost:31188
\end{verbatim}

Ou directement pour utiliser avec les infos de debug de emucas.elf\\
\verb|sh3eb-elf-gdb -i=mi -ex "target remote localhost:31188" emucas.elf|

Pour mettre un point d'arret utiliser \verb|b| ou \verb|hb| (\verb|hbreak|
(hardware break, n\'ecessaire pour le premier point d'arr\^et).
\end{enumerate}

\end{giacjshere}
\end{document}
